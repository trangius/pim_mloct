% Programmering i matematiken - med Matlab & Octave (c)
% by Krister Trangius & Emil Hall
%
% Programmering i matematiken - med Matlab & Octave is licensed under a
% Creative Commons  Attribution-ShareAlike 4.0 International License.
%
% You should have received a copy of the license along with this work. If not,
% see <http://creativecommons.org/licenses/by-sa/4.0/>.
%------------------------------------------------------------------------------

\newcommand{\partintro}[1]{{\LARGE\scshape\color{myBlue} #1}}

%----------------------------------------------------------------------------------------
%  HEADINGS
%----------------------------------------------------------------------------------------
\renewcommand{\chaptername}{Kapitel} % used by headers
\addto\captionsswedish{\renewcommand{\chaptername}{Kapitel}} % ...or rather this is used in headers?
\addto\extrasswedish{\renewcommand{\chapterautorefname}{kapitel}} %Used by autoref
\addto\extrasswedish{\renewcommand{\sectionautorefname}{delkapitel}} %Used by autoref
\addto\extrasswedish{\renewcommand{\subsectionautorefname}{delkapitel}} %Used by autoref
\addto\extrasswedish{\renewcommand{\partautorefname}{del}} %Used by autoref
\addto\extrasswedish{\renewcommand{\tableautorefname}{tabell}} %Used by autoref
\def\lstlistingautorefname{exempel} % Used by autoref
\renewcommand{\figurename}{Figur} % used by headers
\addto\captionsswedish{\renewcommand{\figurename}{Figur}} % ...or rather this is used in headers?
%\renewcommand{\@chapapp}{kapitel} % ... or this, but might as well leave them for now

\titleformat{\chapter}[display]
  {\normalfont\bfseries\color{black}}
  {\filleft%
    \begin{tikzpicture}
	\node[ % the square
      outer sep=0pt,
      text width=2.5cm,
      minimum height=3cm,
      fill=myBlue,
      font=\color{white}\fontsize{80}{90}\selectfont,
      align=center
      ] (num) {\thechapter};
%    \node[ % the rotated "Kapitel"
%      rotate=90,
%      anchor=south,
%      font=\color{black}\Large\normalfont
%      ] at ([xshift=-5pt]num.west) {\textls[180]{\textsc{\chaptertitlename}}};
	 \node[ % this is the title of the chapter (sent as argument #1)
      anchor=south east,
      font=\color{black}\Large\normalfont
      ] at ([yshift=-70pt, xshift=88pt]num.west) {\textls[180]{\textsc{#1}}}; 
\end{tikzpicture} 
  }
  {10pt}
  {\titlerule[2.5pt]\vskip3pt\titlerule\vskip4pt\LARGE\sffamily}

% and here we've got the section:
\makeatletter
\renewcommand{\@seccntformat}[1]{\llap{\textcolor{myBlue}{\csname the#1\endcsname}\hspace{1em}}}                    
\renewcommand{\section}{\@startsection{section}{1}{\z@}
{-2ex \@plus -1ex \@minus -.4ex}
{3pt \@plus.2ex }
{\normalfont\large\sffamily\bfseries}}
\renewcommand{\subsection}{\@startsection {subsection}{2}{\z@}
{-3ex \@plus -0.1ex \@minus -.4ex}
{0.5ex \@plus.2ex }
{\normalfont\sffamily\bfseries}}
\renewcommand{\subsubsection}{\@startsection {subsubsection}{3}{\z@}
{-2ex \@plus -0.1ex \@minus -.2ex}
{.2ex \@plus.2ex }
{\normalfont\small\sffamily\bfseries}}                        
\renewcommand\paragraph{\@startsection{paragraph}{4}{\z@}
{-2ex \@plus-.2ex \@minus .2ex}
{.1ex}
{\normalfont\small\sffamily\bfseries}}


% % Section text styling
% \titlecontents{section}[1.25cm] % Indentation
% {\addvspace{3pt}\sffamily\bfseries} % Spacing and font options for sections
% {\contentslabel[\thecontentslabel]{1.25cm}} % Section number
% {}
% {\hfill\color{black}\thecontentspage} % Page number
% []

% % Subsection text styling
% \titlecontents{subsection}[1.25cm] % Indentation
% {\addvspace{1pt}\sffamily\small} % Spacing and font options for subsections
% {\contentslabel[\thecontentslabel]{1.25cm}} % Subsection number
% {}
% {\ \titlerule*[.5pc]{.}\;\thecontentspage} % Page number
% []
