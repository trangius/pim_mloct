% Programmering i matematiken - med Matlab & Octave (c) by Krister Trangius & Emil Hall
%
% Programmering i matematiken - med Matlab & Octave is licensed under a
% Creative Commons Attribution-ShareAlike 4.0 International License.
%
% You should have received a copy of the license along with this
% work. If not, see <http://creativecommons.org/licenses/by-sa/4.0/>.
% ----------------------------------------------------------------------------------------

\documentclass[14pt]{extbook}
\usepackage[top=2cm,bottom=2cm,left=3cm,right=2.4cm,headsep=24pt,a4paper]{geometry} % Page margins
\input{style.tex}
\input{style-headings.tex}

\fancyfoot[LE, LO]{{\small{Programmering i Matematiken}}}
\fancyfoot[RE, RO]{{\small{\textcopyright \thinspace Krister Trangius, Emil Hall \& Thelin Förlag}}}


% uncomment this to get 13pt:
% \usepackage{type1cm}
% \renewcommand\normalsize{%
%    \@setfontsize\normalsize{13pt}{14.5pt}%
%    \abovedisplayskip 12\p@ \@plus3\p@ \@minus7\p@
%    \abovedisplayshortskip \z@ \@plus3\p@
%    \belowdisplayshortskip 6.5\p@ \@plus3.5\p@ \@minus3\p@
%    \belowdisplayskip \abovedisplayskip
%    \let\@listi\@listI}\normalsize  

\title{Programmering i matematiken - med Matlab ch Octave}
\author{Krister Trangius och Emil Hall}
\date{mars 2018}

% well, these seems to work for printing but gives odd result in pdf viewer TOC:
% \def\frontmatter{%
%     \pagenumbering{roman}
%     \setcounter{page}{1}
% %    \renewcommand{\thesection}{\Roman{section}}
% 	\renewcommand{\thechapter}{\Alph{chapter}}
% }%
% \def\mainmatter{%
%     \pagenumbering{arabic}
%     \setcounter{page}{1}
%     \setcounter{section}{0}
% 	\renewcommand{\thechapter}{\arabic{chapter}}
%     \renewcommand{\thesection}{\arabic{section}}
% }%


\begin{document}
%----------------------------------------------------------------------------------------
%   COPYRIGHT PAGE
%----------------------------------------------------------------------------------------
\newpage
~\vfill
\thispagestyle{empty}

\noindent Copyright \copyright\ 2018-2022 Krister Trangius \&  Emil Hall\\ % Copyright notice
\\
Detta verk är licenserat under\\
Erkännande-DelaLika 4.0 Internationell (CC BY-SA 4.0).\\
\\
Se https://creativecommons.org/licenses/by-sa/4.0/deed.sv\\

\noindent \textsc{Utgiven av Thelin Förlag} % Publisher

Thelin Förlag, Lidköping\\
Tel. 0510-66100, \emph{www.thelinforlag.se}\\
Beställningsnummer J200 4940\\
Tryckeri: JustNu\\
ISBN: 978-91-7379-390-2\\
Foto: Mikael Carlsson

\newpage

%----------------------------------------------------------------------------------------
%   FÖRORD
%----------------------------------------------------------------------------------------
\newpage
\thispagestyle{empty}
{\Large{\textbf{Förord}}}

Hösten 2018 kom programmering in som en del i kurserna Ma1c, Ma2c och Ma3c. Att använda programmering i matematiken kan vara till stor hjälp. Med hjälp av programmering kan vi visualisera sådant som annars ofta upplevs som abstrakt och svårt att få grepp om. Det är också möjligt att göra många beräkningar (och göra dem om och om igen med olika värden) som skulle vara mer eller mindre omöjliga med bara papper och penna.

Samtidigt är det många som har ganska begränsade (eller inga) erfarenheter av programmering när de möter det i matematiken. Syftet med den här boken är främst att du ska få öva dig i att räkna med programmering som hjälp, men vi går också igenom de viktigaste grunderna i programmering, så att du kan göra  vilka matematiska beräkningar som helst.

Programmering är också roligt! Vi hoppas att den här boken ska vara till stor hjälp under din mattekurs och att du, med hjälp av att använda programmering i matematiken, kommer att få ett nytt förhållningssätt till räknande. Ett sätt som kan vara både spännande, utmanande och givande. 

Vi vill tacka Mikael och Rebecka som har varit till stor hjälp vid framtagandet av denna bok. Vi vill också ge ett stort tack till vår käre förläggare Jan-Eric Thelin på Thelin Förlag.
\newline
\newline
Krister Trangius och Emil Hall, april 2018.
\newpage
%----------------------------------------------------------------------------------------
%	TABLE OF CONTENTS
%----------------------------------------------------------------------------------------
%\frontmatter
\pagenumbering{roman}
\renewcommand{\thechapter}{\Alph{chapter}}%
%\pagestyle{empty} % No headers

\setcounter{page}{1}
\tableofcontents % Print the table of contents itself

%\cleardoublepage % Forces the first chapter to start on an odd page so it's on the right
%\clearpage

%----------------------------------------------------------------------------------------
%   0 - no part at all...
%----------------------------------------------------------------------------------------
\pagestyle{fancy} % Print headers again


\input{chaps/om_bok.tex} % ch:ombok
% Programmering i matematiken - med Matlab & Octave (c)
% by Krister Trangius & Emil Hall
%
% Programmering i matematiken - med Matlab & Octave is licensed under a
% Creative Commons  Attribution-ShareAlike 4.0 International License.
%
% You should have received a copy of the license along with this work. If not,
% see <http://creativecommons.org/licenses/by-sa/4.0/>.
%------------------------------------------------------------------------------

\chapter{Matlab och Octave}\label{ch:installation}
I det här kapitlet ska vi översiktligt gå igenom verktygen Matlab och Octave och hur man använder dem. Vilket av dessa verktyg du använder dig av ska inte spela någon roll i den här boken, då samtliga kodexempel är skrivna och testade med båda.
\newpage
%------------MATLAB---------------
\section{Matlab: introduktion}
För att använda Matlab krävs det en licens som kostar pengar. Om du ska använda Matlab, så har förhoppningsvis din skola en licens eller så har du köpt en själv. Om du inte har det, så kan vi rekommendera Octave istället som inte kostar pengar (se \autoref{sec:octave_intro}). Matlab funkar i Microsoft Windows, Mac OS X och Linux.

\subsection{Matlab: kommandofönstret}
När du först startar Matlab så ser det ut ungefär såhär:

\figurec{15cm}{matlab-gui-layout-1.png}{Start-utseendet på Matlab}

Som du ser finns det ett antal olika rutor. Den som kommer vara mest intressant för oss i början av boken är rutan ''Command window''. I den finns två högerpilar:

\begin{matlab}[caption={Tom kommando-rad},label={}]
>>
\end{matlab}

Efter högerpilarna finns din blinkande markör, så du kan skriva text där. Testa att skriva in texten \cw{1+1}, så att det ser ut såhär:

\begin{matlab}[caption={Skrivit in lite matte},label={}]
>> 1+1
\end{matlab}

och tryck sedan på Enter-tangenten. Vad tror du kommer hända?

\begin{matlab}[caption={Hurra, datorn kan räkna!},label={}]
>> 1+1
ans = 2
>>
\end{matlab}

Ordet \cw{ans} är en förkortning av ''answer''.

Notera att \cw{1+1} även hamnar i rutan nere till höger som heter ''Command history''. Det är precis som det låter en lista med alla uträkningar du skrivit in tidigare, sorterade i den ordning du skrev in dem. Prova själv att skriva in flera enkla matteberäkningar i ''Command window'' och se hur de dyker upp i ''Command history''. Om du sedan dubbelklickar på en rad i ''Command history'' så körs denna beräkning igen. Det kanske inte är så viktigt än så länge, men kommer att bli mer användbart längre fram när du vill slippa skriva in en jättelång uträkning två gånger.

\subsection{Filer i Matlab}
I början av boken kommer vi bara behöva använda ''Command window'' men i \autoref{ch:selektion} behöver vi börja arbeta med filer, för att kunna skriva längre kodstycken.

För att skapa en ny fil, tryck i menyn: \emph{File -> New -> Blank M-File}.

Nu har du en tom fil där du kan skriva in samma typ av kommandon som vi tidigare har skrivit in i ''Command window''. Skillnaden är att här körs inte koden direkt efter att du skrivit en rad och tryckt Enter, utan du kan skriva en massa rader och sen köra alltihop på en gång. Bara för att testa detta, skriv in:

\begin{matlab}[caption={Skrivit in lite matte},label={}]
1+1
\end{matlab}

Hitta sedan rätt knapp överst i Editor-rutan. Antingen en grafisk knapp med en grön \emph{Play}-symbol som pekar åt höger, eller i menyn \emph{Debug -> Save file and run}, eller genom att trycka på \emph{F5}-tangenten.

\newpage
%------------OCTAVE---------------

\section{GNU Octave: introduktion}\label{sec:octave_intro}
Om din du eller din skola inte har en licens för Matlab så kan du använda en gratis opensource-klon som heter GNU Octave och funkar ungefär likadant. Octave funkar i Microsoft Windows, Mac OS X, Linux, BSD och en del andra system.

För att kunna programmera i Octave så måste det först finnas nerladdat och installerat på din dator.

\boxlinks{
Octave finns att ladda ner på: \url{https://www.gnu.org/software/octave/}
}

\subsection{GNU Octave: kommandofönstret}
När du först startar Octave så ser det ut ungefär såhär:

\figurec{15cm}{gnu-octave-gui-layout-1.png}{Start-utseendet på GNU Octave}

Som du ser finns det ett antal olika rutor. Det som kommer vara mest intressant för oss i början av boken är rutan ''Command window''. I ''Command window'' finns två högerpilar.

\begin{matlab}[caption={Tom kommando-rad},label={}]
>>
\end{matlab}

Efter högerpilarna finns din blinkande markör, så du kan skriva text där. Skriv in texten \cw{1+1}, så att det ser ut såhär:

\begin{matlab}[caption={Skrivit in lite matte},label={}]
>> 1+1
\end{matlab}

och tryck sedan på Enter-tangenten. Vad tror du kommer hända?

\begin{matlab}[caption={Hurra, datorn kan räkna!},label={}]
>> 1+1
ans = 2
\end{matlab}

\cw{ans} är en förkortning av ''answer''.

Notera att \cw{1+1} även hamnar i rutan nere till vänster som heter ''Command history''. Det är precis som det låter en lista med alla uträkningar du skrivit in tidigare, sorterade i den ordning du skrev in dem. Prova själv att skriva in flera enkla matteberäkningar i ''Command window'' och se hur de dyker upp i ''Command history''. Om du sedan dubbelklickar på en rad i ''Command history'' så körs denna beräkning igen. Det kanske inte är så viktigt än så länge, men kommer att bli mer användbart längre fram när du vill slippa skriva in en jättelång uträkning två gånger.

\subsection{Filer i Octave}
I början av boken kommer vi bara behöva använda ''Command window'' men i \autoref{ch:iteration} behöver vi börja arbeta med filer, för att kunna skriva längre kodstycken.

Längst ner, bredvid ''Command window'' så finns fliken ''Editor'' - klicka på den! I editor-rutan klickar du sedan på File -> New Script.

Nu har du en tom fil där du kan skriva in samma typ av kommandon som vi tidigare har skrivit in i ''Command window''. Skillnaden är att här körs inte koden direkt efter att du skrivit en rad och tryckt Enter, utan du kan skriva en massa rader och sen köra alltihop på en gång. Bara för att testa detta, skriv in:

\begin{matlab}[caption={Skrivit in lite matte},label={}]
1+1
\end{matlab}

Hitta sedan rätt knapp överst i Editor-rutan. Antingen en grafisk knapp med ett kugghjul och en \emph{Play}-symbol som pekar åt höger, eller i menyn \emph{Run -> Save file and run}, eller genom att trycka på \emph{F5}-tangenten.

\figurec{15cm}{gnu-octave-gui-run-file.png}{Knapp för att köra en fil med kod}
\newpage
%------------OCTAVE ONLINE---------------

\section{Octave Online: introduktion}
Om du inte vill/kan ladda ner och installera program så kan du använda en gratisversion i webbläsaren istället.

\boxlinks{
Octave Online finns på: \url{https://octave-online.net/}
}

När du först går in på Octave Online så ser det ut ungefär såhär:

\figurec{15cm}{octave-online-gui-layout-1.png}{Start-utseendet på Octave Online}

Som du ser finns det ett antal olika rutor. De som kommer vara mest intressant för oss i början av boken är de vita rutorna. I det smala vita fältet längst ner (som kallas ''Command Prompt'') finns två högerpilar. Klicka i det fältet!

Efter högerpilarna finns din blinkande markör, så du kan skriva text där. Skriv in texten \cw{1+1} och tryck sedan på Enter-tangenten. Vad tror du kommer hända?

\begin{matlab}[caption={Hurra, datorn kan räkna!},label={}]
ans = 2
\end{matlab}

\cw{ans} är en förkortning av ''answer''.
\newpage

\subsection{Filer i Octave Online}
I början av boken kommer vi bara behöva använda ''Command window'' men i \autoref{ch:iteration} behöver vi börja arbeta med filer, för att kunna skriva längre kodstycken.

För att kunna skapa filer i Octave Online så behöver du ett konto. Det går att skapa ett nytt konto på sidan. Det är också möjligt att logga in med sitt Google-konto.

För att skapa en ny fil, tryck på ''create empty file'' (en ikon) uppe till vänster:

\figurec{8cm}{octave-online-gui-new-file.png}{Knapp för att skapa en ny fil}

Välj vad filen ska heta. När du har klickat på OK dyker filen upp i listan till vänster. Klicka på den nya filen.

Nu har du en tom fil där du kan skriva in samma typ av kommandon som vi tidigare har skrivit in i ''Command Prompt''. Skillnaden är att här körs inte koden direkt efter att du skrivit en rad och tryckt Enter, utan du kan skriva en massa rader och sen köra alltihop på en gång. Bara för att testa detta, skriv in:

\begin{matlab}[caption={Skrivit in lite matte},label={}]
1+1
\end{matlab}

För att kunna köra ditt program, tryck först på spara-knappen, därefter på \emph{run}.

\figurec{13cm}{octave-online-save-run-buttons.png}{Spara filen och kör}
 %ch:installation

%----------------------------------------------------------------------------------------
%	PART 1 - GRUNDERNA I PROGRAMMERING
%----------------------------------------------------------------------------------------
\thispagestyle{plain} % Print headers again
\part{Grunderna i programmering}{img/cover/code-photo-matlab-part-1}

\mainmatter
\renewcommand{\thechapter}{\arabic{chapter}}%
\setcounter{chapter}{0}% Equivalent to "letter O"

\pagestyle{fancy} % Print headers again
% Programmering i matematiken - med Matlab & Octave (c) by Krister Trangius & Emil Hall
%
% Programmering i matematiken - med Matlab & Octave is licensed under a
% Creative Commons Attribution-ShareAlike 4.0 International License.
%
% You should have received a copy of the license along with this
% work. If not, see <http://creativecommons.org/licenses/by-sa/4.0/>.
% ----------------------------------------------------------------------------------------
\chapter{Datorn som miniräknare}\label{ch:datorn_som_raknemaskin}

I det här kapitlet kommer vi att lära oss att använda Matlab/Octave som en miniräknare. Detta är en viktig grund innan vi går vidare till att använda Matlab/Octave för att rita grafer och programmera.


\section{Kommentarer}\index{kommentarer|textbf}

När vi skriver in tal och matteberäkningar i Matlab/Octave så utför datorn dessa beräkningar. Men ibland kan det vara användbart för vår egen skull att skriva in text som datorn \emph{inte} bryr sig om. Som minnesanteckningar till oss själva. Om vi skriver ett procenttecken \cw{\%} så kommer datorn att strunta i resten av raden, vad det än står där. Detta kallas inom programmering för \emph{kommentarer}:

\begin{matlab}[caption={Vår första kommentar},label={}]
>> 1+1 % allt efter procenttecknet är bara en kommentar
ans = 2
\end{matlab}

Våra kodexempel i boken kommer hädanefter ofta att innehålla små kommentarer som förklarar detaljer i koden, när vi inte skriver förklaringarna i brödtexten ovanför eller nedanför kodexemplet.
\newpage
\section{Aritmetik: de fyra räknesätten}\index{aritmetik|textbf}\index{addition|textbf}\index{multiplikation|textbf}\index{subtraktion|textbf}\index{division|textbf}
I föregående kapitel så testade vi kommandofönstret i Matlab/Octave genom att skriva \cw{1+1}. Låt oss testa de fyra räknesätten. Vi börjar med bara heltal (så tar vi decimaltal senare):

\begin{matlab}[caption={De fyra räknesätten},label={}]
>> 1-1
ans = 0
>> 1+5-3
ans = 3
>> 122-300
ans = -178
>> 3*5
ans = 15
>> 20+2*7
ans = 34
>> (20+2)*7
ans = 154
>> 21/7
ans = 3
>> 10+6/2
ans = 13
>> (10+6)/2
ans = 8
\end{matlab}

Som du kanske ser ovan, så gäller de vanliga reglerna för de fyra räknesätten även i Matlab/Octave; Multiplikation och division sker före addition och subtraktion och det är, i vanlig ordning, möjligt att styra detta med parenteser. Som du också ser är multiplikationstecknet en asterisk \cw{*}.


\subsection{Decimaltal}\index{decimaltal|textbf}

Vad händer om vi gör en division som inte går jämnt upp?

\begin{matlab}[caption={Decimaltal},label={}]
>> 17 / 2
ans = 8.500
\end{matlab}

Jo, vi får ett decimaltal med en punkt mellan heltalsdelen och decimalerna. Svensk standard är att skriva kommatecken där, men Matlab/Octave följer engelsk standard där man använder punkt istället. Om vi själva vill skriva in ett decimaltal så bör vi också använda punkt:

\begin{matlab}[caption={Decimaltal skrivs med punkt},label={}]
>> 8.5 * 2
ans = 17
\end{matlab}

Om vi skriver in kommatecken så kan det eventuellt också fungera i vissa situationer, men eftersom kommatecken även har andra betydelser inom programmering så finns det stor risk att Matlab/Octave missförstår och ger oss ett helt annat resultat än vi ville:

\begin{matlab}[caption={Varning för kommatecken},label={}]
>> 8,5 * 2
ans = 8
ans = 10
\end{matlab}


\begin{matteovning}{De fyra räknesätten}{fyraraknesatt}
Testa de fyra räknesätten i Matlab/Octave med olika siffror, både heltal och decimaltal.
\end{matteovning}

\section{Operatorer}\label{sec:operatorer}\index{operatorer|textbf}
Inom programmering talar man om något som kallas \emph{operatorer}. Vi har redan använt några operatorer, nämligen de fyra räknesätten (\cw{+}, \cw{-}, \cw{*}, \cw{/}) men det finns fler som du redan känner till från matematiken (och några som du kanske inte känner igen).

I den här boken kommer vi att arbeta med lite olika operatorer och du kommer att lära dig dem allt eftersom. Några operatorer som vi kan titta på redan nu, är de så kallade \emph{jämförelseoperatorerna}\index{jämförelseoperatorer|textbf}:

En jämförelseoperator används för att jämföra två tal. Vi kan se det som att vi frågar datorn om jämförelsen stämmer (t.ex. ifall ett tal är mindre än ett annat) och datorn svarar ja eller nej. Låt oss testa:

\begin{matlab}[caption={Mindre än-operatorn},label={}]
>> 3 < 17 % är 3 mindre än 17?
ans = 1
\end{matlab}

Det stämmer ju att 3 är mindre än 17. Som du ser, så svarar datorn 1. Det är datorns sätt att säga ''ja''. Om det inte hade stämt, hade datorn svarat 0:

\begin{matlab}[caption={Mindre än-operatorn igen},label={}]
>> 17 < 3 % är 17 mindre än 3?
ans = 0
\end{matlab}
\newpage
Här kan du se de jämförelseoperatorer som finns i Matlab/Octave:
\index{jämförelseoperatorer|textbf}
\begin{center}
\captionof{table}{Jämförelseoperatorer} \label{tab:jamforelseoperatorer}
\begin{tabular}{ l | c }
  \hline
  \emph{Tecken} & \emph{Betydelse} \\
  \hline
  \cw{==} & lika med \\
  \cw{<} & mindre än \\
  \cw{>} & mer än \\
  \cw{<=} & mindre än eller lika med \\
  \cw{>=} & mer än eller lika med \\
  \texttildelow \cw{=} & inte lika med \\
  \hline
\end{tabular}
\end{center}

Notera att jämförelseoperatorn \cw{==} består av två lika med-tecken på rad. Det kanske verkar märkligt, men blir begripligt senare, i \autoref{sec:tilldelningsoperatorn} där vi talar om tilldelningsoperatorn som skrivs med endast ett lika med-tecken.

Låt oss testa några av dessa:

\begin{matlab}[caption={Test av jämförelseoperatorer},label={}]
>> 3 > 17 % är 3 mer än 17?
ans = 0
>> 5 >= 5 % är 5 mer än eller lika med 5?
ans = 1
>> 8 == 8 % är 8 lika med 8?
ans = 1
\end{matlab}

Du kanske undrar vad det här ska vara bra för. Är det inte självklart att 3 är mindre än 17? Varför ska vi fråga datorn om det? Jämförelseoperatorer används oftast tillsammans med selektion (som vi går igenom i \autoref{ch:selektion}) och iteration (som vi går igenom i \autoref{ch:iteration}).

\boxlinks{
Det finns också fler operatorer i Matlab/Octave än de vi går igenom i den här boken. Se här för en lista: \url{https://se.mathworks.com/help/matlab/operators-and-elementary-operations.html}
}

\begin{matteovning}{Jämförelseoperatorer}{jamforelseoperatorer}
Testa själv med samtliga jämförelseoperatorer och lite olika tal till höger och vänster!
\end{matteovning}



\section{Kvadratrot}\index{kvadratrot|textbf}\index{sqrt|textbf}

Vi antar att du redan känner till begreppet ''roten ur'', eller \emph{kvadratrot} som det också kallas. Med papper och penna brukar vi skriva till exempel $\sqrt9 = 3$. Men det finns ingen tangent på datorns tangentbord för att skriva ett sådant rot-tecken. I Matlab/Octave räknar vi istället ut kvadratrötter med ordet \cw{sqrt} - förkortning av engelskans \emph{square root}. Sen direkt efter ordet \cw{sqrt} ska vårt tal stå inom parentes:

\begin{matlab}[caption={Kvadratrot, roten ur 9},label={}]
>> sqrt(9)
ans = 3
\end{matlab}


\section{Funktioner i programmering}\index{funktioner|textbf}

Begreppet \emph{funktion} är kanske något du känner igen från din vanliga mattebok? T.ex. kanske du har hört att $y$ är en funktion av $x$, alltså: $y=f(x)$. Här händer något med $x$ inne i funktionen $f$ och $y$ har värdet av $f(x)$. Inom programmering används det begreppet lite annorlunda - varje funktion har ett visst namn. Vi kan se det som att namnet är en förkortning som representerar ett längre stycke kod.

Det finns en massa inbyggda funktioner i Matlab/Octave. Vi har redan lärt oss en av dem, nämligen \cw{sqrt}, och vi ska strax lära oss några till. Det går även att skapa egna funktioner, men det kommer vi inte lära oss i denna bok. Att skapa egna funktioner gör programmerare nämligen mest för att strukturera stora program, och vi kommer inte att skapa så stora program.

Om du inte riktigt förstår allt detta så är det ingen fara - du kommer kunna använda funktioner ändå.

För att använda en funktion skriver vi alltid dess namn, sedan en inledande parentes, sedan så kallade \emph{argument}\index{argument|textbf}, och sist en avslutande parentes. Till exempel:

\begin{matlab}[caption={Repetition av funktionsanvändning},label={}]
>> sqrt(81)
ans = 9
\end{matlab}

\newpage
Vissa funktioner har fler än ett argument. Här är exempel på funktioner som tar två argument. Argumenten står inom paranteserna och separeras med kommatecken: 
\index{min|textbf}\index{max|textbf}\index{mod|textbf}
\begin{matlab}[caption={Funktioner med två argument},label={ex:funktionermedtvaargument}]
>> min(3.5, 17) % ger det minsta av två tal
ans = 3.5000
>> max(3.5, 17) % ger det största av två tal
ans = 17
>> mod(26, 10) % ger rest efter heltalsdivision
ans = 6
\end{matlab}

Nu förstår du kanske varför det inte går så bra att använda kommatecken för decimaltal?

Notera att ett argument i sin tur kan vara ett resultat av en uträkning, t.ex:

\begin{matlab}[caption={Resultat av uträkning som argument},label={}]
>> min(sqrt(81), 8+2)
ans = 9
\end{matlab}

Och det går förstås att göra hur långa kedjor som helst:

\begin{matlab}[caption={Människor från yttre rymden},label={}]
>> min(sqrt(sqrt(81)*3*3), sqrt((8+2)^2))
ans = 9
\end{matlab}

\begin{matteovning}{Testa funktioner}{testaFunktioner}
Testa att använda alla de ovanstående funktionerna med några olika argument. Testa även att använda funktioner som argument till andra funktioner.
\end{matteovning}


\section{Mer matte}
Det finns förstås många fler funktioner inbyggda i Matlab/Octave. Vi kommer nu att lista andra operatorer och funktioner som brukar användas i mattekurserna Ma1c, Ma2c och Ma3c. Vi kommer också titta på ett par konstanter.

\newpage
\subsection{Potenser}\index{potenser|textbf}

I den rena matematiken brukar vi skriva potenser (''upphöjt till'') med små siffror, till exempel $3^2$. Åter igen finns det inget sätt på datorns tangentbord att skriva sådana små siffror. Istället använder vi tecknet som ser ut som en uppåtpil, ett spetsigt hustak (se nästa sida):

\begin{matlab}[caption={Potenser},label={}]
>> 3^2
ans = 9
\end{matlab}


\subsection{Trigonometri}\index{trigonometri|textbf}\index{pi|textbf}

Det speciella talet pi finns inbyggt i Matlab/Octave. Det finns inget specialtecken $\pi$ utan vi skriver helt enkelt:

\begin{matlab}[caption={pi},label={}]
>> pi
ans = 3.1416
\end{matlab}

De trigonometriska funktionerna sinus, cosinus, tangens och deras släktingar, finns också inbyggda. Låt oss testa sinus-funktionen:

\begin{matlab}[caption={Trigonometri},label={}]
>> sind(45)
ans = 0.70711
\end{matlab}

I följande tabell kan bokstaven \cw{v} mellan parenteserna ersättas med valfritt tal:

\begin{center}
\captionof{table}{Trigonometriska funktioner} \label{tab:trigonometriskafunktioner}
\begin{tabular}{ l | l | l | l }
  \hline
  \emph{Matteformel} & \emph{Funktion i Matlab/Octave} \\
  \hline
  $\sin v$ & \cw{sind(v)} \\
  $\cos v$ & \cw{cosd(v)} \\
  $\tan v$ & \cw{tand(v)} \\
  $\sin^{-1} v$ & \cw{asind(v)} \\
  $\cos^{-1} v$ & \cw{acosd(v)} \\
  $\tan^{-1} v$ & \cw{atand(v)} \\
  \hline
\end{tabular}
\end{center}
\index{sinus|textbf}\index{cosinus|textbf}\index{tangens|textbf}\index{arcus sinus|textbf}\index{arcus cosinus|textbf}\index{arcus tangens|textbf}\index{sind|textbf}\index{cosd|textbf}\index{tand|textbf}\index{asind|textbf}\index{acosd|textbf}\index{atand|textbf}

\boxteknisk{
Bokstaven \cw{d} i slutet av funktionernas namn står för \emph{degrees} på engelska, alltså grader på svenska. I den här boken räknar vi bara med grader. Det finns också ett annat sätt att räkna vinklar inom trigonometrin, nämligen \emph{radianer}. Om du skriver \cw{sin} istället för \cw{sind} så blir det radianer istället.
}

\begin{matteovning}{Trigonometriska funktioner}{trigonometriskaFunktioner}
Testa själv med samtliga trigonometriska funktioner och lite olika tal mellan parenteserna!
\end{matteovning}


\subsection{Logaritmer}\index{logaritmer|textbf}\index{log|textbf}\index{e|textbf}
\begin{matlab}[caption={Logaritmer},label={}]
>> e
ans = 2.7183
>> log10(10^3)
ans = 3
>> log(e^3)
ans = 3
\end{matlab}


\subsection{Avrundning}\index{avrundning|textbf}\index{round|textbf}\index{floor|textbf}\index{ceil|textbf}

Som vanligt ska vårt tal stå inom parentes.

\begin{matlab}[caption={Avrundning},label={}]
>> round(3.6) % avrundar till närmaste heltal
ans = 4
>> floor(3.6) % avrundar nedåt
ans = 3
>> floor(-3.6) % nedåt även för negativa tal. ej mot noll
ans = -4
>> ceil(3.2) % avrundar uppåt
ans = 4
\end{matlab}

\subsection{Slumptal}\index{slumptal|textbf}\index{randi|textbf}
Det är möjligt att slumpa fram tal i Matlab/Octave. Vi kan be om att få ett slumptal från och med 1 till och med ett valfritt tal, t.ex. 6 om vi vill simulera ett tärningsslag:

\begin{matlab}[caption={Slumptal},label={}]
>> randi(6)
ans = 2
>> randi(6)
ans = 1
>> randi(6)
ans = 6
\end{matlab}

Om du vill ha ett slumptal inom ett intervall som \emph{inte} börjar på 1, så behöver du arbeta med listor, vilket vi går igenom i \autoref{sec:listorslumptal}
\subsection{Förvandla negativa tal till positiva (abs)}\label{subsec:abs}\index{abs|textbf}
En annan funktion som kan vara användbar är \cw{abs}, som förvandlar negativa tal till positiva:

\begin{matlab}[caption={abs},label={}]
>> abs(-26) % förvandlar negativa tal till positiva
ans = 26
\end{matlab}

\subsection{Mer då?}
I det här kapitlet gick vi igenom sånt som gör Matlab/Octave till en vanlig miniräknare. Det finns såklart sådant som gör att vi kan använda datorn som en grafräknare också. Det kommer vi att gå igenom i \autoref{ch:grafer} men först ska vi lära oss några viktiga grunder i programmering.
 %ch:datornocharitmetik
% Programmering i matematiken - med Matlab & Octave (c) by Krister Trangius & Emil Hall
%
% Programmering i matematiken - med Matlab & Octave is licensed under a
% Creative Commons Attribution-ShareAlike 4.0 International License.
%
% You should have received a copy of the license along with this
% work. If not, see <http://creativecommons.org/licenses/by-sa/4.0/>.
% ----------------------------------------------------------------------------------------
\chapter{Variabler}\label{ch:variabler}\index{variabler|textbf}

I det här kapitlet ska vi gå igenom variabler och hur de funkar i programmering. Vi kommer också lära oss hur man själv kan styra utskrift på ett tydligare sätt.

Du känner kanske redan till begreppet variabler? Traditionellt inom matematiken talar man ofta om okända variabler som $x$ och $y$ eller $a$ och $b$. Variabler i Matlab/Octave är något liknande, men de används lite annorlunda.

En variabel i programmering ses kanske enklast som en låda, med en etikett på. På etiketten står det ett namn och i lådan ligger det ett tal.
\figurec{6cm}{boxes.png}{Variabler kan ses som lådor med etikett och innehåll}

I matematikböcker är variablers namn oftast bara en bokstav långa, men i programmering brukar namnet vara ett helt ord, vilket hjälper oss att hålla reda på dem då vi har många.

En annan skillnad är att variabler i programmering alltid har ett visst värde. I traditionell matematik kan man t.ex. säga att $a^x*a^y=a^{x+y}$ som en generell regel som gäller för alla värden. I Matlab/Octave så arbetar man inte med generella värden på variablerna, utan de ''innehåller'' alltid ett visst tal.

%------------------------------------------------------------------------------

\section{Att skapa en variabel}
För att skapa en variabel hittar man först på ett namn till den, sen skriver man namnet, likamed-tecken, och värdet. Till exempel:
\begin{matlab}[caption={Skapa variabeln celsius},label={}]
>> celsius=17
\end{matlab}

Det är även okej att ha mellanslag om man tycker att det blir mer lättläst när det är mindre tätt:
\begin{matlab}[caption={Skapa variabeln celsius},label={}]
>> celsius = 17
\end{matlab}

Namnet kan inte innehålla åäö, mellanslag eller andra konstiga specialtecken. Om vi vill att namnet ska bestå av flera ord så kan vi som sagt inte använda mellanslag, utan istället brukar vi separera orden med hjälp av understreck; \cw{min\_egen\_variabel}.

%------------------------------------------------------------------------------

\section{Tilldelningsoperatorn =}\label{sec:tilldelningsoperatorn}\index{tilldelningsoperatorn =|textbf}
Lika med-tecknet \cw{=} som vi använde ovan kallas för \emph{tilldelningsoperatorn}. I \autoref{sec:operatorer} tittade vi på jämförelseoperatorn \cw{==}. Inom programmering skiljer man på \emph{jämförelse} och \emph{tilldelning}. Tilldelningsoperatorn skrivs med bara \emph{ett} lika-med tecken och används, som vi såg ovan, när man vill ge en variabel ett värde. Det är mycket viktigt att man inte blandar ihop tilldelningsoperatorn \cw{=} och jämförelseoperatorn \cw{==}.

Traditionellt inom matematiken har du kanske lärt dig att det inte spelar någon roll på vilken sida ''lika med''-tecknet olika tal står. När man programmerar i Matlab/Octave är det dock annorlunda. Variabeln, som ska få ett värde, står till vänster om tilldelningsoperatorn. Värdet som variabeln ska få, står till höger. Detta är alltså inte okej, men testa gärna själv och se vad som händer:

\begin{matlab}[caption={Man får inte sätta variabelnamnet på fel sida},label={}]
>> 17 = celsius % ej ok!
\end{matlab}

Det som ligger på högersidan om tilldelningsoperatorn händer först. Det innebär att vi kan göra beräkningar på högersidan. När det väl har räknats ut, tilldelas variabeln värdet. Vi kan t.ex. plussa ihop en massa siffror och sedan lägga det i variabeln. Här får variabeln \cw{nr} värdet 110:

\begin{matlab}[caption={Beräkningar görs på högersidan om tilldelningsoperatorn},label={}]
>> nr = 100 + 3 + 7
\end{matlab}

Med andra ord sker det i följande två steg:
\begin{enumerate}
\item Talen 100, 3 och 7 summeras till 110.
\item Variabeln \cw{nr} skapas och tilldelas värdet 110.
\end{enumerate}

%------------------------------------------------------------------------------

\section {Rutan ''Workspace'' i Matlab/Octave}
Dags att prata om en till av rutorna i Matlab/Octave. Om du skrivit in ovanstående kodexempel i ''Command window'' så har du skapat de två variablerna \cw{celsius} och \cw{nr}. Dessa går nu att se i den lilla rutan ''Workspace''.

\figurec{9cm}{gnu-octave-gui-workspace-1.png}{Rutan ''Workspace'' efter att vi skapat två variabler}

Än så länge behöver du bara bry dig om kolumnen ''Name'' och kolumnen ''Value''. Octave visar även några andra kolumner som du inte behöver tänka på än.

Det finns en motsvarande ruta i Octave Online som heter ''Vars''. Där kan man klicka på variablernas namn för att se deras värde.

För att rensa alla variabler som du har skapat kan du skriva kommandot \cw{clear}. Då ser du också att rutan workspace blir tom.

%------------------------------------------------------------------------------

\section{Att läsa en variabels innehåll}
Efter att en variabel har skapats så kan man läsa innehållet i den och göra beräkningar med den. Man kan till exempel addera två variabler med varandra. Här tilldelar vi variabeln \cw{summa} det sammanlagda värdet av \cw{nr1} och \cw{nr2}, alltså 655 (se kod på nästa sida):
\newpage
\begin{matlab}[caption={Addera två variabler},label={}]
>> nr1 = 100
nr1 =  100
>> nr2 = 555
nr2 =  555
>> total = nr1 + nr2
total =  655
\end{matlab}

Men, kanske du undrar nu, sa vi inte nyss att man inte får sätta variabelnamn på högersidan? Nja, det får man visst, om de variablerna redan finns sedan tidigare och därmed faktiskt har ett värde (100 och 555 i exemplet ovan). Vad vi egentligen menade var att den variabel som vi vill tilldela ett värde måste stå på vänstersidan.

Viktigt att notera är att koden körs uppifrån och ned. Så först skapar vi två variabler (på rad 1 och rad 3). Därefter adderar vi dem (på rad 5). Det hade inte gått att göra tvärtom, att addera två variabler innan de skapats.

Efter att man har skapat och använt variabeln, kan man fortsätta att använda den. Man kan t.ex. ändra variabelns värde. I följande kodstycke ändrar vi en variabel från att ha värdet 100 till att ha värdet 555:

\begin{matlab}[caption={Ändra värdet på en variabel},label={}]
>> nr = 100
nr =  100
>> nr = 555
nr =  555
\end{matlab}

\boxteknisk{
Faktum är att ordet \emph{variabel} kommer från latinets \emph{variare}, vilket betyder ''ändra''. Jämför svenskans variera!
}

Man kan öka en variabels värde. Det gör man genom att lägga på variabelns (tidigare) värde till sig själv och addera ett nytt värde. Detta kanske är lite förvirrande om du fortfarande tänker på tilldelningsoperatorn som ett lika med-tecken. Men det finns inget som hindrar att vi använder en variabel i en beräkning på högersidan, och skriver samma variabel på vänstersidan. På första raden i följande kodstycke får variabeln \cw{nr} värdet 100, på andra raden får den värdet 150:

\begin{matlab}[caption={Öka värdet på variabeln},label={}]
>> nr = 100
nr =  100
>> nr = nr + 50
nr =  150
\end{matlab}

Om detta vore en matematisk ekvation så vore det förstås omöjligt. Det finns ju inget tal som är lika med sig självt plus 50.

%------------------------------------------------------------------------------

\section{Styra utskrifterna}

Hittills har vi låtit datorn automatiskt skriva ut resultaten av alla våra beräkningar. Men vi måste inte ha det så alltid.

\subsection{Semikolon}

Om vi skriver ett semikolon (\cw{;}) på slutet av raden (innan eventuell kommentar) så får vi inte någon utskrift av resultatet:

\begin{matlab}[caption={Hejda utskrift med semikolon},label={}]
>> nr1 = 100; % skrivs inte ut...
>> nr2 = 555; % skrivs inte ut...
>> summa = nr1 + nr2 % men denna skrivs ut!
summa =  655
\end{matlab}

Detta kan vara användbart när vi skriver längre program framöver och inte vill bli distraherade av en massa onödiga utskrifter.

\subsection{Skriva ut med disp}\index{disp|textbf}

Ordet \cw{disp} är en förkortning av engelskans \emph{display}. Med \cw{disp} kan vi be datorn skriva ut saker:

\begin{matlab}[caption={Skriv ut på kommando},label={}]
>> nr1 = 100;
>> nr2 = 555;
% vi kan skriva ut text med apostrofer,
% så kallade enkelfnuttar:
>> disp('nu ska vi räkna');
nu ska vi räkna
>> disp(nr1); % vi kan skriva ut en variabels värde,
100
>> disp(nr1+nr2+12); % ...och resultatet av en uträkning
667
\end{matlab}
\newpage
%---------------------------------------------------------------

\subsection{Övningar för variabler}

\begin{matteovning}{Höger eller vänster?}{assignOpDirection}
Om du kör dessa tre rader kod:
\vspace{10pt}
\begin{matlab}
a = 1;
b = 2;
a = b
\end{matlab}

Vad är nu värdet på \cw{a} och \cw{b}? Fundera själv innan du testar och läser facit.
\end{matteovning}

%---------------------------------------------------------------
\begin{matteovning}{Kopplas variabler ihop för all framtid?}{assignmentNotReference}
Om du kör dessa tre rader kod:
\vspace{10pt}
\begin{matlab}
a = 1;
b = a;
a = 2
\end{matlab}

Vad är nu värdet på \cw{b}? Fundera själv innan du testar och läser facit.
\end{matteovning}

%---------------------------------------------------------------
\begin{matteovning}{Summan och medelvärdet av tre tal}{sum3AndAverage}
Låt säga att du redan har dessa tre variabler inmatade i Matlab/Octave:
\vspace{10pt}
\begin{matlab}
a = 23;
b = 45;
c = 67;
\end{matlab}

Skriv en rad kod som beräknar summan av dessa variabler och skriver ut summan. Skriv därefter en till rad kod som skriver ut medelvärdet. Kom ihåg att det är datorn som ska göra uträkningen, inte du!
\end{matteovning}
%---------------------------------------------------------------
\begin{matteovning}{Decimaltal till heltal}{roundFloatToInt}
Låt säga att du redan har denna variabel inmatad i Matlab/Octave:
\vspace{10pt}
\begin{matlab}
a = 11.534;
\end{matlab}

Skriv en rad kod som tar denna variabel, omvandlar decimaltalet till närmsta heltal, och skriver ut det.
\end{matteovning}

%------------------------------------------------------------------------------

 %ch:variabler
% Programmering i matematiken - med Matlab & Octave (c) by Krister Trangius & Emil Hall
%
% Programmering i matematiken - med Matlab & Octave is licensed under a
% Creative Commons Attribution-ShareAlike 4.0 International License.
%
% You should have received a copy of the license along with this
% work. If not, see <http://creativecommons.org/licenses/by-sa/4.0/>.
% ----------------------------------------------------------------------------------------
\chapter{Listor}\label{ch:listor}\index{listor|textbf}\index{array|textbf}\index{vektor|textbf}I det här kapitlet ska vi gå igenom listor och se hur de funkar i programmering. Man kan se listor som en speciell typ av variabler (som vi lärde oss i föregående kapitel). En lista kan, till skillnad från en vanlig variabel, innehålla flera värden som ligger på rad. Listor är användbara då man vill hålla reda på många saker samtidigt.

\boxteknisk{
Den formella beteckningen för en lista är \emph{array} eller \emph{vektor}. Termen vektor kan vara lite förvirrande, då den används på olika sätt i olika sammanhang. Ibland syftar man på en riktning eller position i en rymd (då har den x-, y- och kanske z-koordinater). Termen array används bara på engelska men inte på svenska. I denna bok kommer vi att använda termen lista.
}

Ett konkret exempel på en lista kan vara temperaturer olika dagar. Den första dagen var det 15 grader, den andra 13, den tredje 16 osv. Varje temperatur ligger var för sig lagrad i något som kallas \emph{element}\index{element|textbf}. Vi kan komma åt varje element i en lista med något som kallas för \emph{index}\index{index|textbf}.

Här är en illustration av en lista som innehåller fem olika element. Vi kan se varje elements index:

\begin{figure}[H]
\begin{center}
\caption{En lista}
\begin{tabular}{l|*{5}{c}r}
\emph{index} & \emph{1} & \emph{2} & \emph{3} & \emph{4} & \emph{5} \\
\hline
\emph{}       & 17 & -65 & -20 & 9 & 42  \\
\end{tabular}
\end{center}
\end{figure}

Om vi ska skapa denna lista i Matlab/Octave kan vi skriva så här:

\begin{matlab}[caption={Skapa lista},label={ex:skapavektor1}]
% Skapa listan:
temperature = [];

% Tilldela listans element olika värde genom index:
temperature(1) = 17;
temperature(2) = -65;
temperature(3) = -20;
temperature(4) = 9;
temperature(5) = 42;
\end{matlab}

För att sedan använda den kan vi t.ex. göra så här:

\begin{matlab}[caption={Använda lista},label={ex:lasavektor1}]
% Lägg samman värdet på de olika elementen i listan och
% skriv ut medelvärdet:
summa = temperature(1) + temperature(2) + temperature(3) + temperature(4) + temperature (5);
disp(summa / 5);
\end{matlab}

Då får vi följande utskrift:

\vspace{10pt}
\begin{matlab}
-3.400
\end{matlab}

\section{Indexering av listor}\index{indexering (av listor)|textbf}

I \autoref{ex:skapavektor1} tilldelar vi listan fem olika heltal med hjälp av deras index. Genom en siffra indexeras ett specifikt element i listan. Index är det som står inne i parentesen efter variabelnamnet. Observera att indexeringen börjar på 1. Vi kan använda index både för att skriva och läsa ett specifikt element.

När man arbetar med listor och indexering av dem gäller det dock att se upp! Det är lätt hänt att man försöker att läsa ett element som inte finns. Det kommer bli fel då man kör programmet. Här är ett exempel:

\begin{matlab}[caption={Felaktig indexering av en lista},label={}]
temperature = [];

temperature(1) = 17;
temperature(2) = -65;
temperature(3) = -20;
temperature(4) = 9;
temperature(5) = 42;

% Skriv ut elementen med index 1 och 6
disp('First element: ');
disp(temperature(1));
disp('Sixth element: ');
disp(temperature(6)); % fel!
\end{matlab}

Då får vi följande utskrift:

\vspace{10pt}
\begin{matlab}
First element:
 17
Sixth element:
error: temperature(6): out of bound 5
\end{matlab}

Först skrivs värdet på det första elementet, \cw{temperature(1)} ut. Därefter får vi som väntat ett felmeddelande. Felet är \emph{''temperature(6): out of bound 5''}. Vi försöker helt enkelt indexera ett element som ligger utanför vår lista.

Det går alltså inte att läsa ett index som inte finns, men däremot går det jättebra att skriva till ett index som inte finns. Då kommer det helt enkelt att skapas. Och om vi skriver till ett index som redan finns så kommer det gamla värdet att ersättas, precis som det funkar för vanliga variabler.

\section{Listans längd}\index{size|textbf}

En lista har ett antal element. Antalet kallas även listans längd. När vi skriver till ett element med ett nytt index så växer listan - den blir längre. Vi kan mäta listans längd med hjälp av funktionen \cw{size}. Den används såhär:

\begin{matlab}[caption={Listans längd},label={}]
temperature = [];

% Stoppa in tre element i listan,
% och skriv ut listans längd efter varje steg:
temperature(1) = 17;
disp(size(temperature, 2));
temperature(2) = -65;
disp(size(temperature, 2));
temperature(3) = -20;
disp(size(temperature, 2));
\end{matlab}

Vi får resultatet:

\vspace{10pt}
\begin{matlab}
 1
 2
 3
\end{matlab}

\boxteknisk{
Du kanske undrar varför det står \cw{size(temperature, 2)} och inte bara
\newline
\cw{size(temperature)}, det hade ju känts enklare. Anledningen har med matriser att göra men är mer avancerad än vad vi tar upp i denna bok.
}

Vad händer om vi inte lägger till element i direkt nummerordning? Då skapas nya element upp till och med det index som vi anger. Alla tidigare element får värdet noll:

\begin{matlab}[caption={Listan fylls automatiskt på med tomma element},label={}]
temperature = [];
temperature(3) = -20;
disp(temperature);
\end{matlab}

Vi får resultatet:

\vspace{10pt}
\begin{matlab}
	0    0  -20
\end{matlab}

\section{Listans sista element}\index{end|textbf}

Vår listas första element är alltid \cw{temperature(1)}, men vad är dess sista element? Det sista elementets index varierar ju. Vi kan alltid komma åt det sista elementet genom att mäta längden, såhär:
\vspace{10pt}
\begin{matlab}
temperature(size(temperature, 2))
\end{matlab}
Men eftersom det är långt och tjatigt att skriva så finns det ett kortare sätt:
\vspace{10pt}
\begin{matlab}
temperature(end)
\end{matlab}
Och om vi vill lägga till nya element i slutet av listan, så att den växer, så kan vi skriva såhär:
\vspace{10pt}
\begin{matlab}
temperature(end+1) = 13;
temperature(end+1) = 20;
% nu har vi lagt till två nya element
\end{matlab}
\newpage
\section{Förenklat sätt att skapa listor}\index{zeros|textbf}\index{ones|textbf}
Det finns också ett enklare sätt att skapa listor på:

\begin{matlab}[caption={Förenklat sätt att skapa listor},label={}]
temperature = [17 -65 -20 9 42];
\end{matlab}

Eller om vi vill skapa en lista med ett visst antal element, och alla element ska ha samma värde:

\begin{matlab}[caption={Massproduktion},label={}]
temperature = zeros(1, 4); % samma som [0 0 0 0]
temperature = ones(1, 4); % samma som [1 1 1 1]
\end{matlab}

Detta kan vara praktiskt om man vill skapa väldigt många element och slippa skriva in dem ett i taget. Också praktiskt om det önskade antalet element ligger i en variabel.

\boxteknisk{
Du kanske undrar varför det står \cw{zeros(1, 5)} och inte bara \cw{zeros(5)}, det hade ju känts enklare. Anledningen har åter igen med matriser att göra men är mer avancerad än vad vi tar upp i denna bok.
}

% \section{Att ta bort element ur en lista}

% Man kan ta bort element ur en lista, inte bara lägga till dem. Koden för att ta bort ett element är lite märklig, för den liknar koden för att skapa en ny lista, men det är bara att gilla läget. Här är ett exempel där vi tar bort element med index 3 ur listan:

% \begin{matlab}[caption={Ta bort element ur en lista},label={}]
% temperature = [111 222 333 444 555 666 777];

% % Ta bort det tredje elementet ur listan
% temperature(3) = [];

% % Skriv ut så vi ser skillnaden
% disp(temperature);
% disp(size(temperature, 2));
% \end{matlab}

% Vi får resultatet:

% \begin{matlab}
% 	111 222 444 555 666 777
% 	6
% \end{matlab}

% Notera att när vi tar bort ett element så krymper listans längd, och efterföljande element ''flyttas ner ett snäpp''. Det element som tidigare låg på index 4 kommer att flyttas ner till index 3, och det som tidigare låg på index 5 kommer att flyttas ner till index 4, och så vidare.


\section{Fylla en lista med nummer i ordning}\label{subsec:filllist}

Ofta kommer vi vilja fylla en lista med tal på rad, t.ex. \cw{[-2 -1 0 1 2]}. Då finns det ett koncisare sätt att skriva. Vi kan skriva det lägsta talet, sedan ett kolon \cw{:} och sedan det högsta talet:
\begin{matlab}[caption={Skapa en lista som innehåller alla heltal från -2 till 2},label={}]
x = [-2 : 2];
\end{matlab}

Med hjälp av ett till kolon så kan vi dessutom ta kortare ''steg'' mellan elementen:

\begin{matlab}[caption={Anpassa steglängd mellan element},label={}]
x = [-2 : 0.5 : 2]
\end{matlab}

Vi får resultatet:

\vspace{10pt}
\begin{matlab}
  -2.00000  -1.50000  -1.00000  -0.50000   0.00000   0.50000   1.00000   1.50000   2.00000
\end{matlab}

Med andra ord är alltså regeln \cw{start:steglängd:slut}, eller bara \cw{start:slut} och då blir steglängden automatiskt 1.


\section{Söka i en lista}\label{sec:findInList}

Vi har lärt oss att läsa elementet med ett visst index. Ibland kan vi istället vilja göra samma sak ''baklänges'' och få svar på frågan: På vilket index ligger ett visst element? T.ex. vilken dag var temperaturen 9 grader? Då kan vi använda funktionen \cw{find}:

\vspace{10pt}
\begin{matlab}
temperature = [17 -65 -20 9 42];
find(temperature == 9) % ger resultatet 4
\end{matlab}


\section{Matematiska operationer på varje element i listan}\label{sec:operationerpaenlista}

Vi kan enkelt göra samma matematiska operation på varje element i en lista, t.ex. multiplicera alla element med 2, eller med sig själva. Vi använder de vanliga operatorerna \cw{+ - * /} men med en punkt framför:

\begin{matlab}[caption={Matematiska operationer på varje element i listan},label={}]
x = [1 2 3 4];
y = x .+ 1;
z = x .* 2;
w = x .* x;
disp(y);
disp(z);
disp(w);
\end{matlab}

Vi får resultatet:

\vspace{10pt}
\begin{matlab}
  2 3 4 5
  2 4 6 8
  1 4 9 16
\end{matlab}

\boxteknisk{
Varför behöver vi skriva punkten framför multiplikationstecknet? Anledningen är mer avancerad än vad vi tar upp i denna bok, men har som vanligt med matriser att göra. Om vi glömmer punkten när vi skriver \cw{z = x .* 2} så funkar det ändå, men om vi glömmer punkten när vi skriver \cw{w = x .* x} så får vi felmeddelandet: \cw{operator *: nonconformant arguments (op1 is 1x4, op2 is 1x4)}
}
\newpage
Det är också möjligt att köra funktioner på varje element i en lista.

\begin{matlab}[caption={Köra funktioner på varje element i en lista},label={}]
x = [4 9 16 25];
disp(sqrt(x));
\end{matlab}

Vi får resultatet:

\vspace{10pt}
\begin{matlab}
2 3 4 5
\end{matlab}

\section{Slumptal med hjälp av listor}\label{sec:listorslumptal}\index{randi|textbf}
Om vi vill slumpa ett tal inom ett valfritt intervall, t.ex. 7 och 42, så gör vi såhär:

\vspace{10pt}
\begin{matlab}
randi([7 42]); % ger ett slumptal fr.o.m 7 t.o.m 42
\end{matlab}

%---------------------------------------------------------------

\section{Övningar för listor}

\begin{matteovning}{Olika sätt att skapa samma lista}{skapaListaMed6Element}
Skapa en lista som innehåller 8 element. Det första elementet ska ha värdet 20, det andra ska ha värdet 30, det tredje 40, och så vidare upp till och med 90. Hur många olika sätt kan du komma på för att skapa en sådan lista? Vilket tycker du känns lämpligast?
\end{matteovning}


\begin{matteovning}{\cw{disp} med lista}{dispMedLista}
Tänk dig följande kod:
\vspace{10pt}
\begin{matlab}
temperature = [];
temperature(3) = -20;
temperature(6) = -13;
disp(temperature(1));
disp(size(temperature, 2));
\end{matlab}
Utan att skriva in koden själv i Matlab/Octave, vad tror du att vi får för utskrift?
\end{matteovning}
 %ch:listor
% Programmering i matematiken - med Matlab & Octave (c) by Krister Trangius & Emil Hall
%
% Programmering i matematiken - med Matlab & Octave is licensed under a
% Creative Commons Attribution-ShareAlike 4.0 International License.
%
% You should have received a copy of the license along with this
% work. If not, see <http://creativecommons.org/licenses/by-sa/4.0/>.
% ----------------------------------------------------------------------------------------
\chapter{Grafer och diagram}\label{ch:grafer}\index{grafer|textbf}\index{diagram|textbf}
En stor anledning till att vi programmerar i Matlab/Octave är att det där är väldigt lätt att rita olika sorters grafer och diagram. I det här kapitlet kommer vi gå igenom en funktion som heter \cw{plot} - den använder vi för att rita grafer. Vi kommer också lära oss funktionen \cw{hist} som används för att rita ut histogram.

%==============================================================================

\section{Plot}\index{plot|textbf}

Tänk dig att vi med en termometer har mätt temperaturen en gång per dag, fr.o.m. måndag t.o.m. fredag, så att vi har 5 mätvärden. Om vi har våra mätvärden i en lista så kan vi ge listan till Matlab/Octave med kommandot \cw{plot} och då kommer vi få upp en graf på skärmen:

\begin{matlab}[caption={Vår första graf},label={}]
temperaturPerDag = [1 4 0 2 3];
plot(temperaturPerDag);
\end{matlab}

Du ser resultatet här nedanför. Notera att linjen börjar på 1 längst till vänster, sen går den upp till 4, ner till 0, och så vidare, precis i samma ordning som vår lista. Notera också att den horisontella axeln går från 1 till 5, vilket motsvarar antalet element i vår lista. Den vertikala axeln går från 0 till 4, vilket beror på att den lägsta temperaturen i vår lista är 0 och den högsta är 4.
\newpage
\figurec{9cm}{gnu-octave-first-plot-example.png}{Vår första graf}

\subsection{Rita punkter}

För att göra grafen lite tydligare så kan vi be Matlab/Octave att rita små cirklar vid varje mätpunkt. Det gör vi genom att lägga till \cw{, '-o'} innan slutparentesen efter \cw{plot}:

\begin{matlab}[caption={Vår andra graf},label={}]
temperaturPerDag = [1 4 0 2 3];
plot(temperaturPerDag, '-o');
\end{matlab}

\boxlinks{
Det finns många fler möjligheter att ställa in och anpassa hur grafen ser ut. T.ex. färger, storlek, var de vertikala och horisontella axlarna ska börja och sluta, med mera. Du kan läsa mer om inställningsmöjligheterna på \url{https://se.mathworks.com/help/matlab/ref/plot.html} och \url{https://se.mathworks.com/help/matlab/creating_plots/change-axis-limits-of-graph.html}
}

\subsection{Flera grafer på samma gång}\index{hold|textbf}

Vad händer om vi använder \cw{plot} flera gånger på rad, fast med olika listor? Jo, varje \cw{plot} rensar skärmen från allt som syntes där tidigare, så vi ser bara resultatet av den sista \cw{plot}:en. Men om vi skriver \cw{hold on;} överst i vårt program så går det att se flera resultat i samma graf. Varje \cw{plot}-linje får då en egen färg, för att vi lättare ska kunna se skillnad på dem:
\newpage
\begin{matlab}[caption={Två grafer i en},label={}]
hold on;
plot([1 4 0 2 3], '-o');
plot([4 5 4 3 2], '-o');
\end{matlab}

\figurec{9cm}{gnu-octave-2-plots-in-1.png}{Två grafer i en}

\subsection{Plotta med två listor, i två dimensioner}\label{subsec:plottwodim}

Tänk om vi hade tänkt mäta temperaturen varje dag i en vecka, men vi glömde mäta på torsdag och fredag. Hur vill vi att vår graf ska se ut då? Det vore bra om det syntes i grafen att två dagar saknas i mitten. Vi kan förstås plotta som vanligt...

\begin{matlab}[caption={Syns inte att torsdag och fredag saknas},label={}]
temperaturPerDag = [10 9 9 7 6];
plot(temperaturPerDag, '-o');
\end{matlab}
(Se graf på nästa sida...)
\newpage
\figurec{9cm}{gnu-octave-missing-days-1.png}{Syns inte att torsdag och fredag saknas}

... men då ser det ut som att vi mätte fem dagar på rad. Det vore bättre om vi fick ett ''glapp'' i grafen, så att det inte finns någon punkt på $x=4$ och $x=5$, men att det sedan finns punkter igen på $x=6$ och $x=7$. Kan vi åstadkomma detta i Matlab/Octave? Såklart vi kan! Men då måste vi använda plot på ett nytt sätt. Istället för att bara skicka in en lista så skickar vi in två listor:

\begin{matlab}[caption={Syns att torsdag och fredag saknas},label={}]
dagar = [1 2 3 6 7]; % vi hoppar över 4 och 5
temperaturPerDag = [10 9 9 7 6];
plot(dagar, temperaturPerDag, '-o');
\end{matlab}

\figurec{9cm}{gnu-octave-missing-days-2.png}{Syns att torsdag och fredag saknas}

Nu fick vi det glapp som vi ville ha, hurra!

Notera att vi på detta sätt kan rita vilka former som helst, även linjer som vänder tillbaka, korsar sig själv, osv:

\begin{matlab}[caption={Plotta vilken form som helst},label={}]
plot([1 5 3 2 4.5], [1 2 4 3 3], '-o');
\end{matlab}

\figurec{9cm}{gnu-octave-2d-plot.png}{Kors och tvärs}

Om vi enkelt vill rita en rät linje, så kan vi använda ovanstående metod och bara ange linjens startpunkt och slutpunkt, såhär:  \cw{plot([xstart xend], [ystart yend])}. Exempel:
\begin{matlab}[caption={Rät linje},label={}]
plot([0 10], [7 5]);
\end{matlab}

\subsection{Plotta en funktion av x}
Nu har vi lärt oss allt vi behöver för att kunna rita en graf av en funktion av x. Som vi såg i \autoref{sec:operationerpaenlista} så är det möjligt att köra en funktion på varje element i en lista. Låt oss testa att rita ut en graf av funktionen \cw{sind} några varv: 

\begin{matlab}[caption={Plotta en sinusvåg},label={}]
x = [0 : 10 : 1080]; % fyll lista med x-värden att visa
y = sind(x); % kör funktionen sind på varje element
plot(x, y); % rita ut grafen!
\end{matlab}

Notera att vi också använder det vi lärde oss i \autoref{subsec:filllist} där vi fyllde en lista med alla nummer mellan -2 och 2.

\subsection{Övningar för plot}

\begin{matteovning}{Återskapa kod efter bild}{aterskapaKodEfterBild}
Skriv koden för att rita följande bild:

\figurec{9cm}{exercise-graph-1.png}{}
\end{matteovning}
\newpage
\begin{matteovning}{Plotta given funktion}{plottaGivenFunktion}
Här är några ofullständiga rader kod:
\vspace{10pt}
\begin{matlab}
x = % här ska det stå något
y = sin(x) ./ x;
% här ska det stå kod för att rita ut grafen
\end{matlab}
Ersätt kommentarerna med kod för att rita följande bild:

\figurec{9cm}{exercise-graph-2.png}{}
\end{matteovning}

%==============================================================================

\section{Histogram}\index{histogram|textbf}

Om vi istället har temperaturen från många dagar så kanske vi hellre vill se en slags sammanfattning - hur många dagar var det 3 grader varmt? Hur många dagar var det 4 grader? Då passar det bra med ett så kallat histogram, även känt som stapeldiagram eller stolpdiagram.

\begin{matlab}[caption={Vårt första histogram},label={}]
temperaturPerDag = [1 4 0 2 3 4 3 6 7 8 9 8 7 7 6 5 7 3 2 3 2 2 1 1 2 1 0];
hist(temperaturPerDag, 0:9);
\end{matlab}
(Se histogram på nästa sida...)
\newpage
\figurec{9cm}{gnu-octave-first-histogram-example.png}{Vårt första histogram}

I histogrammet kan vi lätt se att det var 3 grader varmt fyra dagar och 4 grader varmt i två dagar. Funktionen \cw{hist} tar alltså två argument. Det första är en lista av tal att sammanfatta, som kan vara hur lång som helst. Det andra argumentet berättar hur vi vill sammanfatta listan, närmare bestämt vilka staplar vi vill ha. \cw{0:9} betyder att den första stapeln ska vara 0 och den sista stapeln ska vara 9. Prova vad som händer om du ändrar till t.ex. \cw{0:5} eller \cw{-5:15}! (Ser du likheten med \autoref{subsec:filllist} där vi fyllde en vektor med alla nummer mellan -2 och 2?)

När vi vill visa en sån här lista, där vi vet att den lägsta temperaturen är 0 och den högsta är 9, så passar det förstås bäst att ha just de staplarna i diagrammet. Mer generellt, när vi har en lista som bara innehåller heltal (inga decimaltal), och har ett ganska litet antal olika heltal, så passar det bäst att ha en stapel för varje heltal. Vi kan automatisera det såhär:

\begin{matlab}[caption={Välja rätt antal staplar automatiskt},label={}]
temperaturPerDag = [1 4 0 2 3 4 3 6 7 8 9 8 7 7 6 5 7 3 2 3 2 2 1 1 2 1 0];
hist(temperaturPerDag, min(temperaturPerDag):max(temperaturPerDag));
\end{matlab}

Med ovanstående kod kan vi lätt lägga till nya temperaturer utan att behöva komma ihåg att ändra på \cw{hist}-raden.

Men om vi istället har en lista som innehåller väldigt många olika heltal, eller decimaltal, så passar det inte bäst med något särskilt antal staplar utan är mer av en smaksak.

%==============================================================================

 %ch:grafer
% Programmering i matematiken - med Matlab & Octave (c)
% by Krister Trangius & Emil Hall
%
% Programmering i matematiken - med Matlab & Octave is licensed under a
% Creative Commons  Attribution-ShareAlike 4.0 International License.
%
% You should have received a copy of the license along with this work. If not,
% see <http://creativecommons.org/licenses/by-sa/4.0/>.
%------------------------------------------------------------------------------

\chapter{Selektion (med if)}\label{ch:selektion}\index{selektion|textbf}
Ett program behöver ofta göra olika val beroende på olika värden på saker och ting (t.ex. olika variablers värden). Då använder man något som kallas för \emph{selektion}. I det här kapitlet kommer vi att gå igenom \cw{if}-satsen. Den utför selektion men brukar i sig kallas för \emph{villkorssats}.

Hittills har vi bara skrivit in vår kod i ''Command window'' men nu kommer våra program att växa och då blir det mycket enklare om vi använder filer. Om du inte redan är bekant med det, se \autoref{ch:installation} för hur du ska göra med just ditt program (om du använder Matlab, Octave eller Octave Online).

\section{Läsa in variabler}\index{input|textbf}
Innan vi går vidare med selektion så ska vi ta ett kort sidospår och lära oss hur man kan läsa in variabler medan en kodsnutt körs. Det är smidigt om man vill köra samma kodsnutt flera gånger, men testa olika värden på variablerna. Hittills när vi har skrivit kod i ''Command window'' har vi ju bara gjort det för oss själva. Men när vi nu börjar arbeta med filer, så har vi ju möjlighet att låta andra personer köra vår kod.

Vi skapar en ny fil och skriver följande kod:

\begin{matlab}[caption={Läsa in variabler},label={}]
a = input('Ange variabeln a: ');
b = input('Ange variabeln b: ');
disp(a+b);
\end{matlab}

Funktionen \cw{input} skriver alltså ut en text på skärmen, och låter användaren (den som kör vår kod) skriva in något. När användaren tryckt på Enter-tangenten, så fortsätter vår kod köras, och tilldelar det som användaren har skrivit in till variablerna.

\begin{matteovning}{input}{input}
Testa funktionen \cw{input}. Prova lite olika utskrifter och olika namn på variabler.
\newline
\newline
Testa att köra exemplet ovan, men istället för att skriva in en siffra så skriver du in ditt namn. Vad får du för felmeddelande? Vad tror du det beror på?
\end{matteovning}


\section{if-satsen}\index{if|textbf}
En \cw{if}-sats jämför alltså två värden med varandra. En \cw{if}-sats på svenska blir alltså en om-sats:

\begin{pseudo}
OM något SÅ
	gör detta.
SLUT OM
\end{pseudo}

T.ex. så kan man kontrollera hur varmt vatten är:

\begin{pseudo}
OM temperatur är 100 SÅ
	skriv ut "Nu kokar vattnet!" på skärmen.
SLUT OM
\end{pseudo}
Detta kan illustreras visuellt som ett flödesschema som ser ut så här:

\figurec{12cm}{flodesschema/if1.png}{If-sats som flödesschema}

\newpage
Låt oss prova detta i Matlab/Octave. Vi lägger till lite kod för att användaren ska få mata in värdet på variabeln \cw{temperature}:

\begin{matlab}[caption={Vår första if-sats},label={code:kokar1}]
temperature = input('Ange temperatur: ');
if temperature == 100
    disp('Nu kokar vattnet!');
end
\end{matlab}

Du minns väl att det är skillnad på jämförelseoperatorn och tilldelningsoperatorn? Om du inte minns, se \autoref{sec:operatorer} och \autoref{sec:tilldelningsoperatorn}. När vi arbetar med \cw{if}-satser använder vi jämförelseoperatorn just för att jämföra två olika tal (eller i exemplet ovan, värdet av variabeln \cw{temperature} och talet \cw{100}).

Lägg märke till att det inte är något semikolon \cw{;} efter \cw{if}-satsen. I kodblocket, alltså det som ligger efter \cw{if temperature == 100} och innan \cw{end}, ligger den kod som vi vill utföra, ifall villkorssatsen visar sig stämma. I exemplet ovan har vi bara en rad kod att utföra i kodblocket.

Om vi anger att vattnet är 100 grader, får vi alltså följande resultat:

\vspace{10pt}
\begin{matlab}
Ange temperatur: 100
Nu kokar vattnet!
\end{matlab}

\subsection{else}\index{else|textbf}
Att använda en \cw{if}-sats utan något mer, gör att vi kör ett stycke kod om villkorssatsen visar sig stämma. Annars gör vi ingenting speciellt utan programmet fortsätter bara att köra. Låt oss fortsätta med temperaturer. Om vi i körningen av \autoref{code:kokar1} angav något annat än 100, slutar programmet abrupt:

\vspace{10pt}
\begin{matlab}
Ange temperatur: 89
\end{matlab}

Men ofta vill man ju faktiskt göra något annat, om det visar sig att villkorssatsen \emph{inte} stämmer. Låt oss fortsätta med det kokande vattnet, först på svenska:

\begin{pseudo}
OM temperatur är 100 SÅ
   Skriv ut "Nu kokar vattnet!" på skärmen.
ANNARS
   Skriv ut "Vattnet är inte exakt 100 grader..."
SLUT OM
\end{pseudo}
\newpage
Detta kan illustreras som flödesschema så här:

\figurec{12cm}{flodesschema/if2.png}{if och else som flödesschema}

 Låt oss programmera detta i Matlab/Octave:

\begin{matlab}[caption={Vår första else-sats},label={}]
temperature = input('Ange temperatur: ');
if temperature == 100
    disp('Nu kokar vattnet!');
else
    disp('Vattnet är inte exakt 100 grader...');
end
\end{matlab}

Nu får vi i alla fall ett meddelande, om vi anger att temperaturen är annat än 100 grader:

\vspace{10pt}
\begin{matlab}
Ange temperatur: 89
Vattnet är inte exakt 100 grader...
\end{matlab}

\subsection{if-satser med mindre än-operatorn <}\index{jämförelseoperatorer}
Låt oss testa \cw{if}-satser med en annan jämförelseoperator. Vi tar mindre än-operatorn \cw{<}.

Vi tar det först på svenska:

\begin{pseudo}
OM temperatur är mindre än 100 SÅ
   Skriv ut "Vattnet är inte tillräckligt varmt än..." på skärmen.
ANNARS
   Skriv ut "Vattnet kokar!"
\end{pseudo}
\newpage
Kodat i Matlab/Octave blir det:

\begin{matlab}[caption={Mindre än-operatorn},label={}]
temperature = input('Ange temperatur: ');
if temperature < 100
    disp('Vattnet är inte tillräckligt varmt än...');
else
    disp('Vattnet kokar!');
end
\end{matlab}

På samma sätt som med operatorerna \cw{==} och \cw{<}, kan du använda \cw{if}-satser med de övriga jämförelseoperatorerna som finns listade i \autoref{sec:operatorer}.


\subsection{Input med bokstäver}

Ibland känns det lite tråkigt att vi bara kan prata siffror med datorn. Det vore roligare att kunna säga små ord till den, i alla fall ''j'' och ''n'' för att symbolisera ja/nej.
Vi skapar ett program som ställer frågan ''Är det fint väder?''. Om användaren svarar ''j'' skriver programmet ut ''Vi går på picknick!''. Annars händer ingenting. Men hur ska datorn kunna förstå svaret ''j''? Vi kan använda följande trick:

\begin{matlab}[caption={Kontrollera vädret},label={ml:kontrolleraVadret}]
j = 1; % det här är tricket
svaret = input('Är det fint väder? ');
if svaret == j
    disp('Vi går på picknick!');
end
\end{matlab}

%---------------------------------------------------------------

\section{Övningar}

\begin{matteovning}{Kontrollera vädret (fortsättning)}{kontrolleraVadret2}
Arbeta vidare på \autoref{ml:kontrolleraVadret} men lägg till att användaren kan svara ''n''. Då skriver programmet ut ''Vi stannar inne och läser en bok''. Är det klurigt? Fundera på värdet på variabeln \cw{n}.
\end{matteovning}

%---------------------------------------------------------------

\begin{matteovning}{Var är det kallast?}{varArDetKallast}
Skapa ett program där man får mata in temperaturen i Östersund och Göteborg. Programmet ska sedan berätta var det är kallast. Men om det är lika kallt i båda städerna så ska programmet berätta detta istället.
\end{matteovning}

%---------------------------------------------------------------

\begin{matteovning}{Felaktig if-sats}{felaktigIfSats}
Något stämmer inte riktigt med följande if-sats:

\vspace{10pt}
\begin{matlab}
x = 9;
if x = 10
    disp('den är 10!');
end
\end{matlab}

När vi försöker köra koden så får vi ett felmeddelande - vad är det som inte stämmer? Skriv om koden så att det blir rätt!
\end{matteovning}
 %ch:selektion
\input{chaps/iteration.tex} %ch:iteration
\input{chaps/problemlosning.tex} %ch:problemlosning

%----------------------------------------------------------------------------------------
%	PART 2 - EXEMPEL OCH ÖVNINGAR???
%----------------------------------------------------------------------------------------
\thispagestyle{plain} % Print headers again
\part{Övningar och facit}{img/cover/code-photo-matlab-part-2}

\pagestyle{fancy} % Print headers again
\input{chaps/ovningar.tex}
% Programmering i matematiken - med Matlab & Octave (c) by Krister Trangius & Emil Hall
%
% Programmering i matematiken - med Matlab & Octave is licensed under a
% Creative Commons Attribution-ShareAlike 4.0 International License.
%
% You should have received a copy of the license along with this
% work. If not, see <http://creativecommons.org/licenses/by-sa/4.0/>.
% ----------------------------------------------------------------------------------------
\chapter{Facit och lösningsförslag}\label{ch:facit}
Här följer facit till denna bok. I många av övningarna i denna bok ska du skriva kod. Den kod som presenteras här ska ses som lösningsförslag, snarare än ett absolut facit. Med programmering går det ju att lösa ett problem på många olika sätt.
%------------------------------------------------------------------------------

\section*{\facchapref{ch:datorn_som_raknemaskin}}

Övningarna i kapitlet behöver inget facit.

%------------------------------------------------------------------------------

\section*{\facchapref{ch:variabler}}

\subsection*{\facovref{ov:assignOpDirection}}
Efter att ha kört de tre kodraderna så är \cw{a=2} och \cw{b=2}. Alltså, när det står två variabler på varsin sida om tilldelningsoperatorn så ändras den på vänstra sidan.

\subsection*{\facovref{ov:assignmentNotReference}}
På rad 2 så får \cw{b} det dåvarande värdet av \cw{a}, alltså blir \cw{b=1}. Efter att ha kört de tre kodraderna så är \cw{b=1} fortfarande. Det är \emph{inte} så att \cw{b} ''kopplas ihop'' med \cw{a} för all framtid. När vi på rad 3 ändrar värdet på \cw{a} så ändras alltså \emph{inte} värdet på \cw{b}.

\subsection*{\facovref{ov:sum3AndAverage}}
\vspace{3pt}
\begin{matlab}
a = 23;
b = 45;
c = 67;
disp(a + b + c);
disp((a + b + c) / 3);
\end{matlab}

\subsection*{\facovref{ov:roundFloatToInt}}
\vspace{3pt}
\begin{matlab}
a = 11.534;
disp(round(a));
\end{matlab}

\section*{\facchapref{ch:listor}}

\subsection*{\facovref{ov:skapaListaMed6Element}}
\vspace{3pt}
\begin{matlab}
% omständigt sätt 1:
a = [];
a(1) = 20;
a(2) = 30;
a(3) = 40;
a(4) = 50;
a(5) = 60;
a(6) = 70;
a(7) = 80;
a(8) = 90;

% omständigt sätt 2:
a = [];
a(end + 1) = 20;
a(end + 1) = a(end) + 10;
a(end + 1) = a(end) + 10;
a(end + 1) = a(end) + 10;
a(end + 1) = a(end) + 10;
a(end + 1) = a(end) + 10;
a(end + 1) = a(end) + 10;
a(end + 1) = a(end) + 10;

% kortare sätt:
a = [20 30 40 50 60 70 80 90];

% ännu bättre sätt:
a = [20 : 10 : 90];
\end{matlab}

\subsection*{\facovref{ov:dispMedLista}}
Vi får följande utskrift:
\vspace{10pt}
\begin{matlab}
0
6
\end{matlab}


% \subsection*{\facovref{ov:listansLangd}}
% XX
% \subsection*{\facovref{ov:operatorerPaVarjeElement}}
% XX
%------------------------------------------------------------------------------

\section*{\facchapref{ch:grafer}}

\subsection*{\facovref{ov:aterskapaKodEfterBild}}
\vspace{3pt}
\begin{matlab}
hold on;
plot([-3 3], [4 5], '-o');
plot([-3 : 3], [6 7 5 6 5 4 3], '-o');
\end{matlab}

\subsection*{\facovref{ov:plottaGivenFunktion}}
\vspace{3pt}
\begin{matlab}
x = [-20 : 0.1 : 20];
y = sin(x) ./ x;
plot(x, y);
\end{matlab}
Notera att om vi bara använder \cw{x = [-20 : 20]} så blir grafen fult kantig. Därför tar vi kortare steg med steglängden \cw{0.1}.

%------------------------------------------------------------------------------

\section*{\facchapref{ch:selektion}}

\subsection*{\facovref{ov:input}}
Jag testar att skriva in mitt namn som variabeln \cw{b}:
\vspace{10pt}
\begin{matlab}
Ange variabeln a: 12
Ange variabeln b: Emil
error: 'Emil' undefined near line 1 column 1
\end{matlab}
Matlab/Octave säger att det inte existerar någon variabel som heter \cw{Emil}. Du kan prova att först skapa en variabel och sen skriva in dess namn så kommer det att funka.
\newpage
\subsection*{\facovref{ov:kontrolleraVadret2}}
\vspace{3pt}
\begin{matlab}
j = 1;
n = 0;
svaret = input('Är det fint väder? ');
if svaret == j
    disp('Vi går på picknick!');
end
if svaret == n
    disp('Vi stannar inne och läser en bok');
end
\end{matlab}

\subsection*{\facovref{ov:varArDetKallast}}
\vspace{3pt}
\begin{matlab}
ostersund_temp = input('Ange temperaturen i Östersund: ');
goteborg_temp = input('Ange temperaturen i Göteborg: ');
if ostersund_temp < goteborg_temp
    disp('Det är kallast i Östersund');
end
if goteborg_temp < ostersund_temp
    disp('Det är kallast i Göteborg');
end
if goteborg_temp == ostersund_temp
    disp('Det är lika kallt');
end
\end{matlab}

\subsection*{\facovref{ov:felaktigIfSats}}
Felet är att vi har råkat använda tilldelningsoperatorn \cw{=} istället för jämförelseoperatorn \cw{==}. Vi ville förstås egentligen skriva såhär:
\vspace{10pt}
\begin{matlab}
x = 9;
if x == 10
    disp('den är 10!');
end
\end{matlab}


%------------------------------------------------------------------------------

\section*{\facchapref{ch:iteration}}

\subsection*{\facovref{ov:talMellan1Och20}}
\vspace{3pt}
\begin{matlab}
tal = 1;
while tal <= 20
	disp(tal);
	tal = tal + 1;
end
\end{matlab}

\subsection*{\facovref{ov:talMellan1Och100}}
\vspace{3pt}
\begin{matlab}
tal = input('Ange tal: ');
while tal <= 100
	disp(tal);
	tal = tal + 1;
end
\end{matlab}

\subsection*{\facovref{ov:singlaSlant}}
\vspace{3pt}
\begin{matlab}
antal_singlingar = input('Ange hur många gånger du vill singla slant: ');
i = 1;
while i <= antal_singlingar
	if randi(2) == 1
		disp('Krona');
	else
		disp('Klave');
	end
	i = i + 1;
end
\end{matlab}

\subsection*{\facovref{ov:slumpaFemTarningssslag}}
\vspace{3pt}
\begin{matlab}
i = 1;
while i <= 5
	disp(randi(6));
	i = i + 1;
end
\end{matlab}
\newpage
\subsection*{\facovref{ov:vaderstationen}}
\vspace{3pt}
\begin{matlab}
num_temperatures = input('Ange antal mätningar: ');
% skapa tom lista som kommer fyllas på med temperaturer:
all_temperatures = [];
i = 1; % räknare för antal iterationer
total = 0; % används för att räkna ut medelvärdet
while i <= num_temperatures
	temperature = input('Ange temperaturmätning: ');
	% lägg till mätningen i slutet av listan:
	all_temperatures(end + 1) = temperature;
	total = total + temperature;
	i = i + 1;
end
disp(all_temperatures);
% räkna ut medelvärdet:
average = total / num_temperatures;
disp(average);
\end{matlab}

\subsection*{\facovref{ov:multiplikationstabellen}}
\vspace{3pt}
\begin{matlab}
i = 1; % det ena som ska gångras med...
while i <= 10
	j = 1; % ... det andra
	while j <= 10
		disp(i * j);
		j = j + 1;
	end
	i = i + 1;
	disp('---'); % separator
end
\end{matlab}
%------------------------------------------------------------------------------

\section*{\facchapref{ch:problemlosning}}
\subsection*{\facovref{ov:gissatalet}}
\vspace{3pt}
\begin{matlab}
answer = randi(100); % slumpa ett tal mellan 1 och 100

nr_guesses = 1;
guess = input('Gissa ett tal mellan 1 och 100: ');
while guess ~= answer
    if guess < answer
        guess = input('Fel! Mitt tal är högre. Gissa igen: ');
    end
    if guess > answer
        guess = input('Fel! Mitt tal är lägre. Gissa igen: ');
    end
    nr_guesses = nr_guesses + 1;
end
disp('Rätt! Såhär många gissningar behövde du:');
disp(nr_guesses);
\end{matlab}

\subsection*{\facovref{ov:datorngissartalet}}
\vspace{3pt}
\begin{matlab}
r = 1; % rätt
l = 2; % lägre
h = 3; % högre
user_input = 0; % ska hantera r, l eller h
min = 1;
max = 100;

nr_guesses=0;

disp('===============================================');

while user_input ~= r
	% vi avrundar nedåt för att bara jobba med heltal:
    guess = floor((max+min)/2);
    disp('Jag gissar på:');
    disp(guess);
    user_input = input('Är det [r]ätt? Eller är ditt tal [h]ögre eller [l]ägre? ');
    if user_input == h
        min = guess;
    end
    if user_input == l
        max = guess;
    end
	nr_guesses = nr_guesses + 1;
end

disp('Såhär många gissningar behövde jag:');
disp(nr_guesses);
\end{matlab}

%------------------------------------------------------------------------------

\section*{\facchapref{ch:ovningar}}

\subsection*{\facovref{ov:kontrolleraFaktorer}}
\vspace{3pt}
\begin{matlab}
all_factors = [3 5 7 17 23]; % här kan Kim ändra
expected_product = 41055; % här också
product = 1;
i = 1;
while i <= size(all_factors, 2)
	product = product * all_factors(i);
	i = i + 1;
end
if product == expected_product
	disp('Rätt');
else
	disp('Fel');
end
\end{matlab}

\subsection*{\facovref{ov:delbartMedTre}}
\vspace{3pt}
\begin{matlab}
the_number = input('Skriv ett heltal: ');
% blir det ingen rest om vi dividerar talet med 3?
if mod(the_number, 3) == 0
	disp('delbart');
else
	disp('ej delbart');
end
\end{matlab}

\subsection*{\facovref{ov:primtalEllerEj}}
\vspace{3pt}
\begin{matlab}
the_number = input('Skriv ett heltal: ');
is_prime = 1;
factor = 2;
while factor <= sqrt(the_number)
	% blir det ingen rest om vi dividerar?
	% dvs, är den jämt delbar?
	% dvs, är "factor" en faktor i "the_number"?
	if mod(the_number, factor) == 0
		is_prime = 0;
	end
	factor += 1;
end
if is_prime
	disp('primtal');
else
	disp('ej primtal');
end
\end{matlab}

\subsection*{\facovref{ov:faktoriseraTillPrimtalsFaktorer}}
\vspace{3pt}
\begin{matlab}
remaining = input('Skriv ett heltal: ');
factor = 2;
while remaining > 1
	% blir det ingen rest om vi dividerar?
	% dvs, är den jämt delbar?
	% dvs, är "factor" en faktor i "remaining"?
	if mod(remaining, factor) == 0
		% vi har hittat en faktor, skriv ut den
		disp(factor);
		% efter denna faktor, vad blir kvar av talet?
		remaining = remaining / factor;
	else
		factor = factor + 1;
	end
end
\end{matlab}

\subsection*{\facovref{ov:testaFaktoriseringsProgrammet}}
\vspace{3pt}
\begin{matlab}
the_number = input('Skriv ett heltal: ');
% faktorisera:
remaining = the_number;
factor = 2;
all_factors = [];
while remaining > 1
	if mod(remaining, factor) == 0
		disp(factor);
		all_factors(end + 1) = factor;
		remaining = remaining / factor;
	else
		factor = factor + 1;
	end
end
% kontrollräkna:
product = 1;
i = 1;
while i <= size(all_factors, 2)
	product = product * all_factors(i);
	i = i + 1;
end
if product == the_number
	disp('Rätt');
else
	disp('Fel');
end

\end{matlab}


\subsection*{\facovref{ov:funktionenRandi}}
Övningen behöver inget facit.

\subsection*{\facovref{ov:datornLangsam}}
Övningen behöver inget facit.

\subsection*{\facovref{ov:twoDiceSumHistShapeChange}}
När vi bara gör 100 kast så blir histogrammet ofta ganska ojämnt och ''taggigt''. Men ju fler kast, desto mer antar histogrammet formen av en likbent triangel.
\subsection*{\facovref{ov:twoDiceDiffProgram}}
\vspace{3pt}
\begin{matlab}
random_numbers = [];
number_throws = 100; % den här kan man ändra
throw = 1;
while throw <= number_throws;
	dice1 = randi(6);
	dice2 = randi(6);
	difference = abs(dice1 - dice2);
	random_numbers(end + 1) = difference;
	throw = throw + 1;
end
hist(random_numbers, min(random_numbers):max(random_numbers));
\end{matlab}

\subsection*{\facovref{ov:twoDiceDiff}}
Den vanligaste skillnaden mellan två tärningar är 1.

\subsection*{\facovref{ov:myOwnHistogramFunction}}
\vspace{3pt}
\begin{matlab}
% ... detta är fortsättning på koden i övningen ovan
% med andra ord funkar denna kod inte för sig själv
low = min(random_numbers);
high = max(random_numbers);
num_points_in_histogram = 1 + high - low;
summary = zeros(1, num_points_in_histogram);
i = 1;
while i <= size(random_numbers, 2)
	one_number = random_numbers(i);
	index_in_summary = one_number + 1 - low;
	summary(index_in_summary) = summary(index_in_summary) + 1;
	i = i + 1;
end
xaxis = [low : high];
plot(xaxis, summary, '-o');
% gör att y-axeln alltid börjar på 0,
% men välj automatiskt var den slutar:
ylim([0 inf]);
\end{matlab}
\boxlinks{
	Läs mer om \cw{ylim} här: \url{http://se.mathworks.com/help/matlab/creating_plots/change-axis-limits-of-graph.html}
}

\subsection*{\facovref{ov:yatzy1}}
Rätt svar är ca 0,077\% men det krävs väldigt många spelomgångar för att få ett bra närmevärde.
\vspace{10pt}
\begin{matlab}
total_nr_yatzy = 0; % hur många gånger vi fått yatzy
nr_games = 100000; % hur många spel vi vill köra
game = 1;
while game <= nr_games

  % slå en tärning:
  first = randi(6);
  % slå fyra tärningar till, se om alla blir samma:
  all_are_same = 1;
  die = 2; % för vi har redan slagit tärning 1
  while die <=5
    if randi(6) ~= first
      all_are_same = 0;
    end
    die = die + 1; % gå vidare till nästa tärning
  end

  if all_are_same == 1 % blev det yatzy?
    total_nr_yatzy = total_nr_yatzy + 1;
  end
  game = game + 1; % för att gå vidare till nästa spel
end

% skriv ut vårt närmevärde till sannolikheten för yatzy,
% i procent:
disp(100 * total_nr_yatzy / nr_games);
\end{matlab}

\subsection*{\facovref{ov:yatzy2}}
% iterera fem tärningar
% die = 1;
% while die <=5
%   dices(die) = randi(6);
%   die = die + 1; % gå vidare till nästa tärning
% end

% disp('Tärningssslag:');
% disp(dices);

% räkna ut vilket tärningsvärde som vi slog flest av.
% det är de tärningarna som vi vill spara nästa slag.

\vspace{3pt}
\begin{matlab}
dices = [1 4 5 4 3]; % lista med våra fem tärningar

save_eyes = 0; % vilket tärningsvärde vi slog flest av
nr_save = 0; % antal tärningar med det värdet

% iterera sex gånger, en för varje tänkbart tärningsvärde
eyes = 1;
while eyes <= 6

  % räkna hur många tärningar som visar just detta värde
  counter = 0;
  die = 1;
  while die <= 5
    if dices(die) == eyes
        counter = counter + 1;
    end
    die = die + 1;
  end

  % ska vi byta vilket tärningsvärde vi ska spara?
  if nr_save < counter
    nr_save = counter;
    save_eyes = eyes;
  end

  eyes = eyes + 1;
end
disp('Bäst att spara alla tärningar med antal ögon:');
disp(save_eyes);
\end{matlab}

Alternativ lösning:
\vspace{10pt}
\begin{matlab}
dices = [1 4 5 4 3]; % lista med våra fem tärningar
dice_histogram = zeros(1, 6);
die = 1;
while die <= 5
	eyes = dices(die);
	dice_histogram(eyes) = dice_histogram(eyes) + 1;
	die = die + 1;
end
% dice_histogram(1) är hur många 1:or vi slog
% dice_histogram(2) är hur många 2:or vi slog, osv
nr_save = 0;
eyes = 1;
while eyes <= 6
	if nr_save < dice_histogram(eyes)
		nr_save = dice_histogram(eyes);
		save_eyes = eyes;
	end
	eyes = eyes + 1;
end
disp(save_eyes);
\end{matlab}
(Se nästa sida för ytterligare en alternativ lösning)
\newpage
Alternativ lösning för den som vill lära sig mer om hur man kan använda \cw{hist}, \cw{max} och \cw{find}:
\vspace{10pt}
\begin{matlab}
dices = [1 4 5 4 3]; % lista med våra fem tärningar
dice_histogram = hist(dices, 1:6);
save_eyes = find(dice_histogram == max(dice_histogram), 1);
disp(save_eyes);
\end{matlab}

Ännu kortare lösning:
\vspace{10pt}
\begin{matlab}
dices = [1 4 5 4 3]; % lista med våra fem tärningar
save_eyes = mode(dices);
disp(save_eyes);
\end{matlab}

\boxlinks{
	För mer information om \cw{mode}, se: \url{https://se.mathworks.com/help/matlab/ref/mode.html}:
}

\subsection*{\facovref{ov:chansTillYatzy}}
Ungefär 4.6\% chans.
\vspace{10pt}
\begin{matlab}
dices = [0 0 0 0 0]; % lista med våra fem tärningar
total_nr_yatzy = 0; % hur många gånger vi fått yatzy

nr_games = 10000; % hur många spel vi vill köra
game = 1;
while game <= nr_games
  % save_eyes håller reda på vilket tärningsvärde vi har
  % flest av. Här sätter vi den till -1 för vi ska behöva
  % slå om alla tärningar första gången i ett nytt spel:
  save_eyes = -1;

  % roll är en räknare som går från 1 till 3.
  % vi itererar alltså tre gånger, en gång för varje slag:
  roll = 1;
  while roll <= 3

    % iterera fem tärningar, slå om några eller alla:
    die = 1;
    while die <= 5
      % om vi inte vill spara tärningen, slå om den:
      if dices(die) ~= save_eyes
        dices(die) = randi(6);
      end
      die = die + 1; % gå vidare till nästa tärning
    end

    % räkna ut vilket tärningsvärde som vi slog flest av.
    % det är de tärningarna som vi vill spara nästa slag.

    save_eyes = 0; % vilket tärningsvärde vi slog flest av
    nr_save = 0; % antal tärningar med det värdet

    % iterera 6 gånger, ett varv för varje
    % tänkbart tärningsvärde
    eyes = 1;
    while eyes <= 6

      % räkna hur många tärningar som visar just detta
      % värde
      counter = 0;
      die = 1;
      while die <= 5
        if dices(die) == eyes
          counter = counter + 1;
        end
        die = die + 1;
      end

      % ska vi byta vilket tärningsvärde vi ska spara?
      if counter > nr_save
        nr_save = counter;
        save_eyes = eyes;
      end

      eyes = eyes + 1;
    end

    roll = roll + 1; % för att gå vidare till nästa slag
  end
  if nr_save == 5 % blev det yatzy?
    total_nr_yatzy = total_nr_yatzy + 1;
  end
  game = game + 1; % för att gå vidare till nästa spel
end

% skriv ut vårt närmevärde till sannolikheten för yatzy,
% i procent
disp(100 * total_nr_yatzy / nr_games);
\end{matlab}

\newpage
\subsection*{\facovref{ov:standardavvikelse}}
Medelvärde: 6.1500, standardavvikelse: 3.8010
\vspace{10pt}
\begin{matlab}
my_list = [4 9 10 7.5 8 9 3 9 4 -2];
nr_items = size(my_list, 2);

% räkna ut summan för alla tal i listan:
total = 0;
i = 1;
while i <= nr_items
    total = total + my_list(i);
    i = i + 1;
end

% räkna ut medelvärdet:
mid = total / nr_items;

% räkna ut standardavvikelse:
total = 0; % återställ total till 0
i = 1;
while i <= nr_items
    total = total + (my_list(i) - mid)^2;
    i = i + 1;
end

std_dev = sqrt(total/ (nr_items-1));
% nr_items-1 pga Bessels korrektion

disp(mid);
disp(std_dev);
\end{matlab}

\subsection*{\facovref{ov:typvarde}}
Typvärde: 9
\vspace{10pt}
\begin{matlab}
my_list = [4 9 10 7.5 8 9 3 9 4 -2];
nr_items = size(my_list, 2);

nr_most_common = 0; % antal element med typvärdet

i = 1;
while i <= nr_items

	% räkna antal element med "det här" värdet:
	counter = 0;
	j = 1;
	while j <= nr_items
		if my_list(j) == my_list(i)
			counter = counter + 1;
		end
		j = j + 1;
	end

	% ska vi ersätta det förra typvärdet?
	if nr_most_common < counter
		nr_most_common = counter;
		most_common = my_list(i);
	end

	i = i + 1;
end
disp('Typvärde: ');
disp(most_common);
\end{matlab}


\subsection*{\facovref{ov:standardavvikelse2}}
\vspace{3pt}
\begin{matlab}
my_list = [4 9 10 7.5 8 9 3 9 4 -2];
disp(mean(my_list)); % medelvärde
disp(mode(my_list)); % typvärde
disp(std(my_list)); % standardavvikelse
\end{matlab}

%==============================================================================

\subsection*{\facovref{ov:enkelNumeriskLosning}}
\vspace{3pt}
\begin{matlab}
x = -10;
while x <= 10
	if 3 * x - 7 == 5
		disp(x);
	end
	x = x + 1;
end
\end{matlab}

\subsection*{\facovref{ov:enkelNumeriskLosningFel1}}
Vi ändrar på rad 3 i föregående kodstycke till \cw{if 3 .* x .- 7 == 29}. Programmet misslyckas för att lösningen då är 12, vilket ligger utanför intervallet -10 till 10 som ju är de värden vi testar. Den lättaste fixen är att testa ett större intervall, till exempel -1000 till 1000.

\subsection*{\facovref{ov:enkelNumeriskLosningFel2}}
Vi ändrar på rad 3 i föregående kodstycke till \cw{if 3 .* x .- 7 == 4}. Lösningen på $3x-7=4$ är x=3,6666... men programmet kan bara hitta heltalslösningar. När datorn testar \cw{x=3} så blir vänsterledet lika med 2, och när datorn i nästa varv testar \cw{x=4} så blir vänsterledet lika med 5. Vänsterledet ''hoppar'' alltså direkt från 2 till 5, och hoppar över rätt svar. Det finns inget jättelätt sätt att fixa programmet - vi behöver byta till en mer avancerad metod.

\subsection*{\facovref{ov:testaSolve}}
$x = -6$
\vspace{10pt}
\begin{matlab}
syms x;
solve(4*x+15==-9, x)
\end{matlab}

\subsection*{\facovref{ov:intervallhalvering}}\index{intervallhalvering|textbf}\index{binär sökning|textbf}
Precis som i ''gissa talet''-spelet så är det smart att gissa mitt emellan två tal. Sedan testar vi vår gissning \cw{xmed} genom att mata in den i ekvationen, och kolla om vi hamnade \emph{under} eller \emph{över} $4$. Den grundläggande strategin brukar kallas \emph{binär sökning} eller \emph{intervallhalvering}, och används som lösning på en mängd olika problem inom programmering.
\vspace{10pt}
\begin{matlab}
xmin = -10; % dessa måste sättas manuellt,
xmax = 10; % så att de "omfamnar" svaret
i = 1;
while i <= 20 % ju fler iterationer desto mer exakt svar
	xmed = (xmin + xmax) / 2; % gissa mitt emellan
	% testa ekvationens (snarare olikhetens) värde
	% i tre punkter:
	ymin = 3 * xmin - 7 < 4;
	ymax = 3 * xmax - 7 < 4;
	ymed = 3 * xmed - 7 < 4;
	% kontrollera att vi är på vardera sidan om rätt svar:
	if ymin == ymax
		disp('fel! välj bättre xmin och xmax. Starta om');
	    i = 10000; % avbryt loopen i förtid
	end
	% välj vilken halva vi ska söka vidare i:
	if ymin ~= ymed
		xmax = xmed;
	end
	if ymax ~= ymed
		xmin = xmed;
	end
	i = i + 1;
end
disp(xmin);
disp(xmax);
\end{matlab}
Det här programmet är för övrigt ett bra exempel på när det skulle löna sig att skapa våra egna funktioner i Matlab/Octave. Med en egen funktion skulle vi kunna slippa upprepa nästan samma ekvations-kod tre gånger. Om du har lust och tid över, sök information om det på nätet, t.ex. på \url{https://se.mathworks.com/help/matlab/ref/function.html}


\subsection*{\facovref{ov:testaSolve2ndDegree}}
$x_{1}=-6$, $x_{2}=1$
\vspace{10pt}
\begin{matlab}
syms x;
solve(x*x + 5*x - 6 == 0, x)
\end{matlab}

\subsection*{\facovref{ov:testaRoots2ndDegree}}
\begin{itemize}
\item \cw{roots([6 -13 5])} ger $x_{1}=1.66667$ och $x_{2}=0.5$
\item \cw{roots([4 -2 -6])} ger $x_{1}=1.5$ och $x_{2}=-1$
\item $2/x - 18 = 0$ är inte en polynomekvation, går inte att lösa med \cw{roots}
\item \cw{roots([3 -2 0])} ger $x_{1}=0.66667$ och $x_{2}=0$
\end{itemize}
\newpage
\subsection*{\facovref{ov:intervallhalvering2}}
Samma program som i \autoref{ov:intervallhalvering}, förutom att vi ändrar startvärdena \cw{xmin} och \cw{xmax}, samt ändrar ekvationen (olikheten). Programmet hittar lösningen $x = 14.667$
\vspace{10pt}
\begin{matlab}
xmin = -50;
xmax = 50;
i = 1;
while i <= 20
	xmed = (xmin + xmax) / 2;
	ymin = xmin^6 - sin(xmin) - 3^xmin + 7 < 0;
	ymax = xmax^6 - sin(xmax) - 3^xmax + 7 < 0;
	ymed = xmed^6 - sin(xmed) - 3^xmed + 7 < 0;
	if ymin == ymax
		disp('fel! välj bättre xmin och xmax. Starta om');
	    i = 10000; % avbryt loopen i förtid
	end
	if ymin ~= ymed
		xmax = xmed;
	end
	if ymax ~= ymed
		xmin = xmed;
	end
	i = i + 1;
end
disp(xmin);
disp(xmax);
\end{matlab}

\subsection*{\facovref{ov:intervallhalvering3}}
Nej, lösningsförslaget i den här boken klarar inte att lösa den ekvationen. Intervallhalverings-metoden bygger ju på att \cw{ymin} och \cw{ymax} ska vara på varsin sida rätt svar, så metoden kräver alltså att kurvan korsar linjen $y=0$. Men polynomfunktionen $y = x^2 + 5x - 6$ bara nuddar vid linjen utan att korsa den! Det finns dock andra numeriska metoder som klarar att lösa sådana ekvationer.
%==============================================================================



\subsection*{\facovref{ov:lisasLillaTunna}}
Radie=1 dm, höjd=2 dm
\vspace{10pt}
\begin{matlab}
radius = [0.5 : 0.1 : 1.5];
volume = 6.2832;
height = volume ./ (pi .* (radius.^2));
area = 2 .* pi .* radius .* (radius .+ height);
plot(radius, area, '-o');
\end{matlab}
\figurec{9cm}{exercise-lisas-lilla-tunna-area.png}{Grafen för ovanstående kod}

%==============================================================================

% \subsection*{\facovref{ov:arsrantaOchManadsranta}}
% \begin{matlab}
% yearly = 3 / 100; % omvandla från procent
% monthly = (1 + yearly)^(1/12) - 1;
% disp(monthly * 100); % skriv ut i procent
% yearly = (1 + monthly)^12 - 1;
% disp(yearly * 100); % skriv ut i procent
% \end{matlab}

\subsection*{\facovref{ov:invanareEfterYYears}}
6727.5 biljetter (troligen avrundat uppåt eller nedåt).
\vspace{10pt}
\begin{matlab}
tickets = [1000]; % ett element per år
yr = 1;
while yr <= 20
	tickets(yr+1) = tickets(yr) * (1 + 10/100);
	yr = yr + 1;
end
disp(tickets(end));
plot(tickets, '-o');
\end{matlab}

\subsection*{\facovref{ov:invanareHurMangaYears}}
Den 18:e festivalen.
\vspace{10pt}
\begin{matlab}
tickets = [1000]; % ett element per år
yr = 1;
while tickets(end) < 5000
	tickets(yr+1) = tickets(yr) * (1 + 10/100);
	yr = yr + 1;
end
disp(yr);
\end{matlab}

\newpage
\subsection*{\facovref{ov:oregelbundenExpFunkt}}
Den 26:e festivalen har 8110.5 biljetter (troligen avrundat uppåt eller nedåt).
\vspace{10pt}
\begin{matlab}
nr_years = 25
changes = [];
% changes innehåller ökningen i procent från föregående festival
% sätt de två första ökningarna:
changes(2) = 4;
changes(3) = 2;
% räkna ut alla andra ökningar enligt mönstret:
yr = 3;
while yr <= nr_years
	changes(yr+1) = changes(yr-1) + 1;
	yr = yr + 1;
end

tickets = [1000]; % ett element per år
yr = 1;
while yr <= nr_years
	change = (1 + changes(yr+1) / 100); 
	tickets(yr+1) = tickets(yr) * change;
	yr = yr + 1;
end
disp(tickets(end));
plot(tickets, '-o');
\end{matlab}


% \subsection*{\facovref{ov:invanareVilkenPercentOkning}}
% XX
% \begin{matlab}
% min_percent = 0;
% max_percent = 100;
% start_tickets = 1024;
% end_tickets = 3125;
% total_years = 5;
% tickets = -1;
% while round(tickets) ~= end_tickets
% 	guess_percent = (min_percent + max_percent) / 2; % gissa mitt emellan

%   tickets = start_tickets;
%   years_passed = 0;
%   while years_passed < total_years
%     tickets = tickets * (1 + guess_percent/100);
%     years_passed = years_passed + 1;
%   end
%   %disp(tickets);
%   disp([tickets, min_percent, guess_percent, max_percent]);

% 	% välj vilken halva vi ska söka vidare i:
% 	if tickets > end_tickets
% 		max_percent = guess_percent;
% 	end
%   if tickets < end_tickets
% 		min_percent = guess_percent;
% 	end
% end
% disp(guess_percent);

% disp(((end_tickets / start_tickets) ^ (1 / total_years)) - 1);
% \end{matlab}
% 25%
% 1024.0   1280.0   1600.0   2000.0   2500.0   3125.0

\subsection*{\facovref{ov:oregelbundenExpFunkt2}}
Ungefär 1300 kaniner.
\vspace{10pt}
\begin{matlab}
hold on;
estimates = [];

i = 1;
while i <= 100
	rabbits = [20]; % ett element per månad
	mon = 1;
	while mon <= 24
		change = 1 + randi([10 28]) / 100;
		rabbits(mon+1) = rabbits(mon) * change;
		mon = mon + 1;
	end

	plot(rabbits);
	estimates(end + 1) = rabbits(end);
	i = i + 1;
end

% skriv ut medelvärde
disp('Ca antal kaniner: ');
disp(mean(estimates));
\end{matlab}

\figurec{9cm}{exercise-rabbits.png}{}

\subsection*{\facovref{ov:avbetalningAvAnnuitetsLan}}
\vspace{3pt}
\begin{matlab}
% konstanter:
interest = 5; % månadsränta i procent
amort_per_month = 1087.07;

% listor:
months = [0];
sum_payed = [0];
months_twice = [0];
debt = [15000];

while debt(end) > 0
  month = months(end) + 1;

  months(end+1) = month;
  sum_payed(end+1) = sum_payed(end) + amort_per_month;

  months_twice(end+1) = month;
  debt(end+1) = debt(end) * (1 + interest / 100);
  months_twice(end+1) = month;
  debt(end+1) = debt(end) - amort_per_month;
end

hold on;
plot(months, sum_payed, '-o');
plot(months_twice, debt);
\end{matlab}

%==============================================================================

\subsection*{\facovref{ov:numeriskDeriveringTabell1}}

Med hjälp av följande kod:
\vspace{10pt}
\begin{matlab}
x = [2007 2008 2009 2010 2011 2012 2013 2014 2015 2016 2017];
y = [383.79 385.60 387.43 389.90 391.65 393.85 396.52 398.65 400.83 404.21 406.53];

index = 10; % det 10e mätvärdet i listan
result = (y(index+1) - y(index-1)) / 2;
disp(result);
\end{matlab}

Får vi resultatet \emph{2.8500} ppm/år.

\subsection*{\facovref{ov:numeriskDeriveringTabell2}}
\vspace{3pt}
\begin{matlab}
x = [2007 2008 2009 2010 2011 2012 2013 2014 2015 2016 2017];
y = [383.79 385.60 387.43 389.90 391.65 393.85 396.52 398.65 400.83 404.21 406.53];

year = input('ange år mellan 2008 och 2016: ');

% hitta index för årtalet användaren vill se:
index = find(x == year);

% räkna ut den centrala differenskvoten kring det året
result = (y(index+1) - y(index-1)) / 2;
disp(result);
\end{matlab}


\subsection*{\facovref{ov:numeriskDeriveringFunktion}}
Ett närmevärde till derivatan är: -0.19048
\vspace{10pt}
\begin{matlab}
x = input('ange x-värde att beräkna derivatan för: ');
h = 0.4;
x_left  = x - h;
x_right = x + h;
y_left  = (x_left^2 + 1) / x_left;
y_right = (x_right^2 + 1) / x_right;
estimate = (y_right - y_left) / (2 * h);
disp('ett närmevärde till derivatan är: ');
disp(estimate);
\end{matlab}


\subsection*{\facovref{ov:numeriskDeriveringOlikaH}}
Följande kod:
\vspace{10pt}
\begin{matlab}
estimates = [];
x = input('ange x-värde att beräkna derivatan för: ');
h = 0.4;
num_halvings = 1;
while num_halvings <= 8
	x_left  = x - h;
	x_right = x + h;
	y_left  = (x_left^2 + 1) / x_left;
	y_right = (x_right^2 + 1) / x_right;
	estimate = (y_right - y_left) / (2 * h);
	estimates(end + 1) = estimate;
	h = h / 2;
	num_halvings = num_halvings + 1;
end
plot(estimates, '-o');
\end{matlab}

Ger grafen:
\figurec{9cm}{exercise-estimate-derivative.png}{}

I grafen ser det ut som att den exakta derivatan är: 0
\newpage
\subsection*{\facovref{ov:testaGradient}}
Den symbolhanterande metoden ger svaret i form av ett bråk: $f'(1) = -5/2$,
medan följande numeriska kod ger approximationen $f'(1) = -2.5$
\vspace{10pt}
\begin{matlab}
xp = 1; % i vilken punkt vill vi veta lutningen
h = 0.001;
x = [xp - h : h : xp + h];
y = 3 * sqrt(x) + 2 ./ x.^2; % vår funktion
estimate = gradient(y, h)(2);
disp(estimate);
\end{matlab}

\subsection*{\facovref{ov:approximeraTaletE}}
Följande kod ger approximationen $e = 2.7183$
\vspace{10pt}
\begin{matlab}
emin = 0; % dessa måste sättas manuellt,
emax = 10; % så att de "omfamnar" svaret.

% för ändringskvot. mindre h ger exaktare svar:
h = 0.00001; 
i = 1;
while i <= 40 % ju fler iterationer desto exaktare svar
	emed = (emin + emax) / 2; % gissa mitt emellan
	% beräkna ändringskvoten kring emed^1:
	slope = (emed^(1+h) - emed^(1-h)) / (2 * h);
	% vi önskar att 'slope' ska bli exakt lika med emed,
	% så välj vilken halva vi ska söka vidare i:
	if slope > emed
		% vår gissning var för hög, sök i nedre halvan
		emax = emed;
	else
		% vår gissning var för låg, sök i övre halvan
		emin = emed;
	end
	i = i + 1;
end
if abs(slope - emed) < 0.001
	disp('Talet e är ungefär = ');
	disp(emed);
else
	disp('Hittade inte något tal e mellan emin och emax.');
	disp('Ändra startvärden och försök igen.');
end
\end{matlab}


\printindex

\end{document}
